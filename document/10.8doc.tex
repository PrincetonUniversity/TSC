\documentstyle[makeidx,psbox,11pt]{report}
%define a command "\setw" to set the width of parbox
%-----------------------------------------------------------
\newlength{\wspace}
\newlength{\width}
\newcommand{\setw}[1]{ \setlength{\width}{\textwidth} \settowidth{\wspace}{#1}
\addtolength{\width}{-\wspace}}
%----------------------------------------------------------
\makeindex
\author{William Daughton \\  Stephen C. Jardin \\  Charles Kessel \\  Neil Pomphrey}
\title{TSC Users Manual-Chapter 4 \\ TSC Usage}
\date{January 2003}
\begin{document}
 \setcounter{chapter}{3}
%\maketitle
\chapter{Input File}
\subsection{Summary of Input File}
\renewcommand{\baselinestretch}{0.56} \large \normalsize
\begin{tabbing}
XXX \= plasma pres\= XXXXX \= XXXXXX \= XXXXXX \= XXXXXX \= XXXXXX \= XXXXXX \= XXXXXX \kill
 \>  \> 11 \> 21 \> 31 \> 41 \> 51 \> 61 \> 71 \\
  \\
NAME CARD\\
\tiny 00 \> \tiny Control \> \tiny IRST1 \>\tiny IRST2 \>\tiny IPEST
\>\tiny NCYCLE \>\tiny  NSKIPR \>\tiny NSKIPL \> \tiny IMOVIE\\
\tiny 01 \> \tiny Dimensions \> \tiny NX \>\tiny NZ \>\tiny ALX
\>\tiny ALZ \>\tiny ISYM \>\tiny CCON \>\tiny IDATA \\
\tiny 02 \> \tiny Time step \> \tiny DTMINS \>\tiny DTMAXS \>\tiny
DTFAC \>\tiny LRSWTCH \>\tiny IDENS \>\tiny IPRES \>\tiny IFUNC
\\
\tiny 03 \> \tiny Numerical \> \tiny XLIM \>\tiny ZLIM \>\tiny
XLIM2 \>\tiny FFAC \>\tiny NDIV \>\tiny ICIRC \>\tiny ISVD \\
\tiny04 \> \tiny Surf. Ave. \> \tiny ISURF \>\tiny NPSI \>\tiny
NSKIPSF \>\tiny TFMULT \>\tiny  ALPHAR \>\tiny BETAR \>\tiny
ITRMOD \\
\tiny 05 \> \tiny Limiter \> \tiny I \>\tiny XLIMA(I) \>\tiny ZLIMA(I)
\>\tiny XLIMA(I+1) \>\tiny ZLIMA(I+1) \>\tiny XLIMA(I+2) \>\tiny
ZLIM(I+2) \\
\tiny 06 \> \tiny Divertor \> \tiny IDIV \>\tiny PSIRAT \>\tiny X1SEP
\>\tiny X2SEP \>\tiny Z1SEP \>\tiny Z2SEP \>\tiny NSEPMAX \\
\tiny 07 \> \tiny Impurities \> \tiny IIMP \>\tiny ILTE \>\tiny
IMPBND \>\tiny IMPPEL \>\tiny AMGAS \>\tiny ZGAS \>\tiny NTHE\\
\tiny 08 \> \tiny Obs. pairs \> \tiny J \>\tiny XOBS(2J-1) \>\tiny
ZOBS(2J-1) \>\tiny XOBS(2J) \>\tiny ZOBS(2J) \> \tiny NPLOTOBS \\
\tiny 09 \> \tiny Ext. coils \> \tiny N \>\tiny XCOIL(N) \>\tiny
ZCOIL(N) \>\tiny IGROUPC(N) \>\tiny ATURNSC(N) \>\tiny RSCOILS(N)
\>\tiny AINDC(N) \\
\tiny 10 \> \tiny Int. coils \> \tiny M \>\tiny XWIRE(M) \>\tiny
ZWIRE(M) \>\tiny IGROUPW(M) \>\tiny ATURNSW(M) \>\tiny RSWIRES(M)
\>\tiny CWICS(M) \\
\tiny 11 \> \tiny ACOEF  \> \tiny ICO \>\tiny NCO \>\tiny
ACOEF(ICO) \>\tiny $\ldots$(ICO+1) \> $\ldots$ \> $\ldots$ \> \tiny $\ldots$(ICO+4)
\\
\tiny 12 \> \tiny Tranport \> \tiny TEVV \>\tiny DCGS \>\tiny QSAW
\>\tiny ZEFF \>\tiny IALPHA \>\tiny IBALSW \>\tiny ISAW  \\
\tiny 13 \> \tiny Init. cond-1 \> \tiny ALPHAG \>\tiny ALPHAP
\>\tiny NEQMAX \>\tiny XPLAS \>\tiny ZPLAS \>\tiny GZERO
\>\tiny QZERO \\
\tiny 14 \> \tiny Init. cond-2 \> \tiny ISTART \>\tiny XZERIC \>\tiny
AXIC \>\tiny ZZERIC \>\tiny BZIC \\
\tiny 15 \> \tiny Coil groups \> \tiny IGROUP \>\tiny GCUR(1)  \> $\ldots$
\> $\ldots$ \> $\ldots$ \> $\ldots$ \>\tiny GCUR(6)\\
\tiny 16 \>\tiny Plasma curr. \> \tiny  -  \>\tiny PCUR(1)  \> $\ldots$ \>
$\ldots$ \> $\ldots$  \> $\ldots$ \>\tiny PCUR(6)\\
\tiny 17 \> \tiny Plasma press. \> \tiny  -  \>\tiny PPRES(1)  \>\tiny
$\ldots$ \> $\ldots$ \> $\ldots$  \> $\ldots$ \>\tiny PPRES(6)\\
\tiny 18 \> \tiny Timing \> \tiny  -  \>\tiny TPRO(1)  \> $\ldots$ \> $\ldots$
\> $\ldots$  \> $\ldots$ \> \tiny TPRO(6)\\
\tiny 19 \> \tiny Feedback-1 \> \tiny  L  \>\tiny NRFB(L)  \>\tiny
NFEEDO(L) \>\tiny FBFAC(L) \>\tiny FBCON(L)  \>\tiny IDELAY(L)
\>\tiny FBFACI(L)\\
\tiny 20 \> \tiny Feedback-2 \> \tiny  L  \>\tiny TFBONS(L)  \>\tiny
TFBOFS(L) \>\tiny FBFAC1(L) \>\tiny FBFACD(L) \>\tiny IPEXT(L)\\
\tiny 21 \> \tiny Contour plot \> \tiny  ICPLET  \>\tiny ICPLGF 
\>\tiny ICPLWF \>\tiny ICPLPR \>\tiny ICPLBV \>\tiny ICPLUV
\>\tiny ICPLXP\\
\tiny 22 \> \tiny Vector plot \> \tiny  IVPLBP  \>\tiny IVPLVI  \>\tiny
IVPLFR \>\tiny IVPLJP \>\tiny IVPLVC \>\tiny IVPLVT \>\tiny -\\
\tiny 23 \> \tiny Aux. heat \> \tiny  -  \>\tiny BEAMP(1)  \> $\ldots$ \>
$\ldots$ \> $\ldots$ \> $\ldots$ \>\tiny BEAMP(6)\\
\tiny 24 \> \tiny Density \> \tiny  -  \>\tiny RNORM(1)  \> $\ldots$ \>
$\ldots$ \> $\ldots$ \> $\ldots$ \> \tiny RNORM(6)\\
\tiny 25 \> \tiny Dep. prof. \> \tiny  ABEAM  \>\tiny DBEAM  \>\tiny
NEBEAM \>\tiny EBEAMKEV \>\tiny AMBEAM  \>\tiny FRACPAR
\>\tiny IBOOTST\\
\tiny 26 \> \tiny Anom. trans. \> \tiny  - \>\tiny FBCHIA(1)  \> $\ldots$ \>
$\ldots$ \> $\ldots$ \> $\ldots$ \> \tiny FBCHIA(6)\\
\tiny 27 \> \tiny Tor. field \> \tiny  - \>\tiny GZEROV(1)  \> $\ldots$ \>
$\ldots$ \> $\ldots$ \> $\ldots$ \>\tiny GZEROV(6)\\
\tiny 28 \> \tiny Loop volt. \> \tiny  - \>\tiny VLOOPV(1)  \> $\ldots$ \>
$\ldots$ \> $\ldots$ \> $\ldots$ \>\tiny VLOOPV(6)\\
\tiny 29 \>  \tiny PEST output\> \tiny  - \>\tiny TPEST(1)  \> $\ldots$ \>
$\ldots$ \> $\ldots$ \> $\ldots$ \>\tiny TPEST(6)\\
\tiny 30 \>\tiny Mag. Axis(x) \> \tiny  - \>\tiny XMAGO(1) \> $\ldots$ \>
$\ldots$ \> $\ldots$ \> $\ldots$ \>\tiny XMAGO(6)\\
\tiny 31 \> \tiny Mag. Axis(z) \> \tiny  - \>\tiny ZMAGO(1) \> $\ldots$ \>
$\ldots$ \> $\ldots$ \> $\ldots$ \>\tiny ZMAGO(6)\\
\tiny 32 \> \tiny Divertor \> \tiny N \>\tiny XLPLATE(N) \>\tiny
ZLPLATE(N) \>\tiny XRPLATE(N) \>\tiny ZRPLATE(N) \>\tiny FPLATE(N,1)
\>\tiny FPLATE(N,2)\\
\tiny 33 \> \tiny Coil grp-2 \>\tiny IGROUP \>\tiny RESGS( )\\
\tiny 34 \> \tiny TEVV(t) \> \tiny  -  \>\tiny TEVVO(1)  \> $\ldots$ \>
$\ldots$ \> $\ldots$ \> $\ldots$ \>\tiny TEVVO(6)\\
\tiny 35 \> \tiny FFAC(t) \> \tiny  -  \>\tiny FFACO(1)  \> $\ldots$ \>
$\ldots$ \> $\ldots$ \> $\ldots$ \>\tiny FFACO(6)\\
\tiny 36 \> \tiny ZEFF(t) \> \tiny  -  \>\tiny ZEFFV(1)  \> $\ldots$ \>
$\ldots$ \> $\ldots$ \> $\ldots$ \>\tiny ZEFFV(6)\\
\tiny 37 \> \tiny Volt group \> \tiny  IGROUP \>\tiny GVOLT(1) \>$\ldots$
\>$\ldots$ \>$\ldots$ \>$\ldots$ \>\tiny GVOLT(6)\\
\tiny 38 \> \tiny LHDEP \> \tiny  ILHCD \>\tiny VILIM  \>\tiny
FREQLH \>\tiny AION \>\tiny ZION \>\tiny CPROF \>\tiny IFK\\
\tiny 39 \> \tiny Ext. coil-2 \> \tiny  N \>\tiny DXCOIL(N)  \>\tiny
DZCOIL(N) \>\tiny FCU(N) \>\tiny FSS(N) \>\tiny TEMPC(N) \>\tiny CCICS(N) \\
\tiny 40 \> \tiny Noplot \> \tiny NOPLOT(1) \>$\ldots$ \>$\ldots$ \>$\ldots$
\>$\ldots$ \>$\ldots$ \>\tiny NOPLOT(7)\\
\tiny 41 \> \tiny Ripple \> \tiny  IRIPPL \>\tiny NTFCOIL \>\tiny
RIPMAX \>\tiny RTFCOIL \>\tiny NPITCH \>\tiny RIPMULT \>\tiny
IRIPMOD\\
\tiny 42 \> \tiny Major rad. \> \tiny - \>\tiny  RZERV(1) \>$\ldots$
\>$\ldots$ \>$\ldots$ \>$\ldots$ \>\tiny RZERV(6) \\
\tiny 43 \> \tiny Minor rad. \> \tiny - \>\tiny  AZERV(1) \>$\ldots$
\>$\ldots$ \>$\ldots$ \>$\ldots$ \>\tiny AZERV(6) \\
\tiny 44 \> \tiny Ellipticity \> \tiny - \>\tiny EZERV(1) \>$\ldots$ \>$\ldots$
\>$\ldots$ \>$\ldots$ \>\tiny EZERV(6) \\
\tiny 45 \> \tiny Triangularity \> \tiny - \>\tiny DZERV(1)\>$\ldots$
\>$\ldots$ \>$\ldots$ \>$\ldots$ \>\tiny DZERV(6) \\
\tiny 46 \> \tiny LH heating \> \tiny - \>\tiny  PLHAMP(1) \>$\ldots$
\>$\ldots$ \>$\ldots$ \>$\ldots$ \>\tiny PLHAMP(6) \\
\tiny 47 \> \tiny Dens. exp-1 \> \tiny - \>\tiny  ALPHARV(1) \> $\ldots$
\> $\ldots$ \> $\ldots$ \> $\ldots$ \>\tiny ALPHARV(6) \\
\tiny 48 \> \tiny Dens. exp-2 \>\tiny - \>\tiny  BETARV(1) \> $\ldots$ \>
$\ldots$ \> $\ldots$ \> $\ldots$ \>\tiny BETARV(6) \\
\tiny 49 \> \tiny Multipole \> \tiny N \>\tiny MULTN(N) \>\tiny
ROMULT(N) \>\tiny IGROUPM(N) \> \tiny ATURNSM(N)\\
\tiny 50 \> \tiny CD \> \tiny - \>\tiny  FRACPAR(1) \> $\ldots$ \>$\ldots$
\>$\ldots$ \>$\ldots$ \>\tiny FRACPAR(6) \\
\tiny 51 \> \tiny alh \> -\>\tiny A(1) \> $\ldots$ \> $\ldots$ \> $\ldots$ \> $\ldots$
\> \tiny A(6)\\
\tiny 52 \> \tiny dlh \> -\>\tiny D(1) \> $\ldots$ \> $\ldots$ \> $\ldots$ \>
$\ldots$ \>\tiny D(6)\\
\tiny 53 \> \tiny a1lh \> -\>\tiny A1(1) \> $\ldots$ \> $\ldots$ \> $\ldots$ \>
$\ldots$ \>\tiny A1(6)\\
\tiny 54 \> \tiny a2lh \> -\>\tiny A2(1) \> $\ldots$ \> $\ldots$ \> $\ldots$ \>
$\ldots$ \>\tiny A2(6)\\
\tiny 55 \> \tiny ac \> -\>\tiny AC(1) \> $\ldots$ \> $\ldots$ \> $\ldots$ \>
$\ldots$ \>\tiny AC(6)\\
\tiny 56 \> \tiny dc \> -\>\tiny DC(1) \> $\ldots$ \> $\ldots$ \> $\ldots$ \>
$\ldots$ \>\tiny DC(6)\\
\tiny 57 \> \tiny ac1 \> -\>\tiny AC1(1) \> $\ldots$ \> $\ldots$ \> $\ldots$ \>
$\ldots$ \>\tiny AC1(6)\\
\tiny 58 \> \tiny ac2 \> -\>\tiny AC2(1) \> $\ldots$ \> $\ldots$ \> $\ldots$ \>
$\ldots$ \>\tiny AC2(6)\\
\tiny 59 \> \tiny ICRH \> -\>\tiny PICRH(1) \> $\ldots$ \> $\ldots$ \> $\ldots$
\> $\ldots$ \> \tiny PICRH(6)\\
\tiny 60 \> \tiny Halo Temp \> - \> \tiny TH(1) \> $\ldots$ \> $\ldots$ \> $\ldots$ \>
$\ldots$ \>\tiny TH(6)\\
\tiny 61 \> \tiny Halo Width \> - \> \tiny AH(1) \> $\ldots$ \> $\ldots$ \> $\ldots$ \>
$\ldots$ \> \tiny AH(6)\\
\tiny 62 \> \tiny X-Shape point \> - \> \tiny XCON0(1) \> $\ldots$ \> $\ldots$ \> $\ldots$ \>
$\ldots$ \> \tiny XCON0(6)\\
\tiny 63 \> \tiny Z-Shape point \> - \> \tiny ZCON0(1) \> $\ldots$ \> $\ldots$ \> $\ldots$ \>
$\ldots$ \> \tiny ZCON0(6)\\
\tiny 64 \> \tiny Fast Wave J \> - \> \tiny FWCD(1) \> $\ldots$ \> $\ldots$ \> $\ldots$ \>
$\ldots$ \> \tiny FWCD(6)\\
\tiny 65 \> \tiny ICRH power profile \>   \> \tiny A(1) \> $\ldots$ \> $\ldots$ \> $\ldots$ \>
$\ldots$ \> \tiny A(6)\\
\tiny 66 \> \tiny ICRH power profile \>   \> \tiny D(1) \> $\ldots$ \> $\ldots$ \> $\ldots$ \>
$\ldots$ \> \tiny D(6)\\
\tiny 67 \> \tiny ICRH power profile \>   \> \tiny A1(1) \> $\ldots$ \> $\ldots$ \> $\ldots$ \>
$\ldots$ \> \tiny A1(6)\\
\tiny 68 \> \tiny ICRH power profile \>   \> \tiny A2(1) \> $\ldots$ \> $\ldots$ \> $\ldots$ \>
$\ldots$ \> \tiny A2(6)\\
\tiny 69 \> \tiny ICRH current profile \>   \> \tiny A(1) \> $\ldots$ \> $\ldots$ \> $\ldots$ \>
$\ldots$ \> \tiny A(6)\\
\tiny 70 \> \tiny ICRH current profile \>   \> \tiny D(1) \> $\ldots$ \> $\ldots$ \> $\ldots$ \>
$\ldots$ \> \tiny D(6)\\
\tiny 71 \> \tiny ICRH current profile \>   \> \tiny A1(1) \> $\ldots$ \> $\ldots$ \> $\ldots$ \>
$\ldots$ \> \tiny A1(6)\\
\tiny 72 \> \tiny ICRH current profile \>   \> \tiny A2(1) \> $\ldots$ \> $\ldots$ \> $\ldots$ \>
$\ldots$ \> \tiny A2(6)\\
\tiny 73 \> \tiny He conf. time \> - \> \tiny HEACT(1) \> $\ldots$ \> $\ldots$ \> $\ldots$ \>
$\ldots$ \> \tiny HEACT(6)\\
\tiny 74 \> \tiny UFILE output \> - \> \tiny TUFILE(1) \> $\ldots$ \> $\ldots$ \> $\ldots$ \>
$\ldots$ \> \tiny TUFILE(6)\\
\tiny 75 \> \tiny Sawtooth time \> - \> \tiny SAWTIME(1) \> $\ldots$ \> $\ldots$ \> $\ldots$ \>
$\ldots$ \> \tiny SAWTIME(6)\\
\tiny 76 \> \tiny Anom. ion trans. \> \tiny  - \>\tiny FBCHIIA(1)  \> $\ldots$ \>
$\ldots$ \> $\ldots$ \> $\ldots$ \> \tiny FBCHIIA(6)\\
\tiny 77 \> \tiny acoef(123) \> - \> \tiny qadd(1) \> $\ldots$ \> $\ldots$ \> $\ldots$ \>
$\ldots$ \> \tiny qadd(6)\\
\tiny 78 \> \tiny acoef(3003) \> - \> \tiny fhmodei(1) \> $\ldots$ \> $\ldots$ \> $\ldots$ \>
$\ldots$ \> \tiny fhmodei(6)\\
\tiny 79 \> \tiny acoef(3011) \> - \> \tiny pwidthc(1) \> $\ldots$ \> $\ldots$ \> $\ldots$ \>
$\ldots$ \> \tiny pwidthc(6)\\
\tiny 80 \> \tiny acoef(3006) \> - \> \tiny chiped(1) \> $\ldots$ \> $\ldots$ \> $\ldots$ \>
$\ldots$ \> \tiny chiped(6)\\
\tiny 81 \> \tiny acoef(3102) \> - \> \tiny tped(1) \> $\ldots$ \> $\ldots$ \> $\ldots$ \>
$\ldots$ \> \tiny tped(6)\\
\tiny 82 \> \tiny impurity fraction \> \tiny imptype \> \tiny frac(1) \> $\ldots$ \> $\ldots$ \> $\ldots$ \>
$\ldots$ \> \tiny frac(6)\\ 
\tiny 83 \> \tiny acoef(3012) \> - \> \tiny nflag(1) \> $\ldots$ \> $\ldots$ \> $\ldots$ \>
$\ldots$ \> \tiny nflag(6)\\
\tiny 84 \> \tiny acoef(3013) \> - \> \tiny expn1(1) \> $\ldots$ \> $\ldots$ \> $\ldots$ \>
$\ldots$ \> \tiny expn1(6)\\
\tiny 85 \> \tiny acoef(3014) \> - \> \tiny expn2(1) \> $\ldots$ \> $\ldots$ \> $\ldots$ \>
$\ldots$ \> \tiny expn2(6)\\
\tiny 86 \> \tiny acoef(3004) \> - \> \tiny firitb(1) \> $\ldots$ \> $\ldots$ \> $\ldots$ \>
$\ldots$ \> \tiny firitb(6)\\
\tiny 87 \> \tiny acoef(3005) \> - \> \tiny secitb(1) \> $\ldots$ \> $\ldots$ \> $\ldots$ \>
$\ldots$ \> \tiny secitb(6)\\
\tiny 88 \> \tiny acoef(881) \> - \> \tiny fracno(1) \> $\ldots$ \> $\ldots$ \> $\ldots$ \>
$\ldots$ \> \tiny fracno(6)\\
\tiny 89 \> \tiny acoef(889) \> - \> \tiny newden(1) \> $\ldots$ \> $\ldots$ \> $\ldots$ \>
$\ldots$ \> \tiny newden(6)\\
\tiny 90 \> \tiny ECRH Power (MW) \>  \>\tiny PECRH(1) \> $\ldots$ \> $\ldots$ \> $\ldots$
\> $\ldots$ \> \tiny PECRH(6)\\
\tiny 91 \> \tiny ECCD Toroidal Current (MA) \>  \>\tiny ECCD(1) \> $\ldots$ \> $\ldots$ \> $\ldots$
\> $\ldots$ \> \tiny ECCD(6)\\
\tiny 92 \> \tiny Sh. Par. “a” (ECCD H\&CD) \>  \>\tiny AECD(1) \> $\ldots$ \> $\ldots$ \> $\ldots$
\> $\ldots$ \> \tiny AECD(6)\\
\tiny 93 \> \tiny Sh. Par. “d” (ECCD H\&CD) \>  \>\tiny DECD(1) \> $\ldots$ \> $\ldots$ \> $\ldots$
\> $\ldots$ \> \tiny DECD(6)\\
\tiny 94 \> \tiny Sh. Par. “a1” (ECCD H\&CD) \>  \>\tiny A1ECD(1) \> $\ldots$ \> $\ldots$ \> $\ldots$
\> $\ldots$ \> \tiny A1ECD(6)\\
\tiny 95 \> \tiny Sh. Par. “a2” (ECCD H\&CD) \>  \>\tiny A2ECD(1) \> $\ldots$ \> $\ldots$ \> $\ldots$
\> $\ldots$ \> \tiny A2ECD(6)\\
\tiny 99 \> 
\end{tabbing}
\renewcommand{\baselinestretch}{1.0} \large \normalsize
\newpage
\subsection{Detailed Description of Input Cards} 
\label{sec:card}
\subsubsection{Card 00 - Control} 
\begin{tabbing} 
XXXXXX \= XXXXXX \= XXXXXX \= XXXXXX \= XXXXXX \= XXXXXX \= XXXXXX
\kill  
11 \> 21 \> 31 \> 41 \> 51 \> 61 \> 71 \\  
\footnotesize IRST1 \>\footnotesize IRST2 \>\footnotesize IPEST \>\footnotesize NCYCLE
\>\footnotesize  NSKIPR \>\footnotesize
NSKIPL \>\footnotesize IMOVIE 
\end{tabbing} 
\setw{NCYCLE = 1.0X }
\begin{tabbing} 
NCYCLE \= = 1.0X \= \parbox[t]{\width}{Last cycle to be computed.  If NCYCLE=0, only the
initial equilibrium is computed.} \kill 
IRST1 \> = 0.0 \> \parbox[t]{\width}{ Start run beginning at time $t$=TPRO(ISTART)} \\ 
          \> = 1.0 \> \parbox[t]{\width}{ Restart run which reads the file SPRSINA.  A restart job\index{restart file}
normally requires only 3 input cards: title, type 00, and type 99.  Cards which specify the evolution
of parameters in time can also be included in a restart job.  However, only the fields corresponding to 
future time points should be changed.} \\ 
          \> = 2.0 \> \parbox[t]{\width}{ Start run which reads initial equilibrium from the file
EQFLINA} \\ 
IRST2 \> = 0.0 \> \parbox[t]{\width}{ Don't write a restart file} \\ 
          \> = 1.0 \> \parbox[t]{\width}{ Do write a restart file.  Restart files are updated every NSKIPL cycles
and at the times specified on the type 29 card and at the end of the run.} \\ 
IPEST \> = 0.0 \> \parbox[t]{\width}{ Don't write PEST file} \index{PEST file}\\    
           \> = 1.0 \> \parbox[t]{\width}{ Do write PEST files EQDSKA and EQDSKASCI at times specified on type
29 cards and at the end of the run. These files can then be read by J-SOLVER code with
IFUNC2=4. (Note:  also writes GEQDSK file compatible with EFIT output.} \\ 
NCYCLE \>      \> \parbox[t]{\width}{ Last cycle to be computed.  If NCYCLE=0, only the
initial equilibrium is computed. Use NCYCLE=-1 to check dimensionless gain for original TSC
control model as described in section~(\ref{sec:control2}).}\\
NSKIPR \>      \> \parbox[t]{\width}{ Number of cycles between print cycles and between times when profile
information is plotted} \\ 
NSKIPL \>      \> \parbox[t]{\width}{ Number of cycles between plot cycles.  Restart files are also written
every NSKIPL cycles.} \\ 
IMOVIE \> = 0.0 \> \parbox[t]{\width}{ Regular graphics} \index{output!graphics}\\
             \> = 1.0 \> \parbox[t]{\width}{ b/w movie with plasma currents plotted} \\ 
             \>  = 3.0 \> \parbox[t]{\width}{ Color movie with plasma currents plotted} \\ 
             \> = 6.0 \> \parbox[t]{\width}{ Color movie of poloidal flux contours and plasma current with
fixed flux increments.  For this option, the data statement in the subroutine CPLOT must be
changed: \\
 \\
PSISMAL-(minimum value)\\
PSINCR-(increment)\\
YMAX \& YMIN at end of CPLOT (now PCUR(12)) \\
} \\ 
             \> = 7.0 \> \parbox[t]{\width}{ Color movie of heat flux distribution on the divertor
plate} \\
             \> = 8.0 \> \parbox[t]{\width}{ Special disruption plots} \\
             \> = 10.   \> \parbox[t]{\width}{ Writes special movie.cdf 
            file for AVS postprocessing.  Note that tmovie and dtmovie are 
            input via the acoef(33) and acoef(34).  Also note that min and
            max range for the major radius plots are given by acoef(42) and
            acoef(43).}
\end{tabbing}
{\bf Note} : All jobs write a equilibrium file EQFLOUA upon termination.  A start job with
IRST1=2 can have ISYM=0 (type 01 card) even if the job which created the equilibrium file had
ISYM=1, as long as the zone {\em size} (dimension of grid spacing) is the same.
\newpage \subsubsection{Card 01 - Dimensions} 
\index{dimensions}
\begin{tabbing} 
XXXXXX \= XXXXXX \= XXXXXX \= XXXXXX \= XXXXXX \= XXXXXX \=
XXXXXX \kill  
11 \> 21 \> 31 \> 41 \> 51 \> 61 \> 71 \\ 
\footnotesize NX \>\footnotesize NZ \>\footnotesize ALX \>\footnotesize ALZ \>\footnotesize
ISYM \>\footnotesize CCON \>\footnotesize IDATA 
\end{tabbing}
\setw{IDATA = 0.0X } 
\begin{tabbing} 
IDATA \= =  0.0X \= \parbox[t]{\width}{Regular run} \kill 
NX  \>  \> \parbox[t]{\width}{Number of zone vertices in the $x$ direction,(NX-1) zones.  Must
have NX$\leq$ PNX-2 (see param.i) (46)} \\ 
NZ  \>  \> \parbox[t]{\width}{ Number of zone vertices in the $z$ direction,(NZ-1) zones.  This
should be an odd number if ISYM = 0. Must have NZ$\leq$ PNZ-2 (see param.i) (16)} \\ 
ALX \>  \> \parbox[t]{\width}{Major radius of outside computational grid boundary in meters
(3.0)}
\\ 
ALZ \>  \> \parbox[t]{\width}{ One half the height of the computational grid boundary in meters
(1.0)}
\\ 
ISYM \> = 0.0 \> \parbox[t]{\width}{ No symmetry about the midplane} \index{symmetry} \\ 
         \> = 1.0 \> \parbox[t]{\width}{ Symmetry about the midplane} \\ 
CCON \>       \> \parbox[t]{\width}{ Major radius of inside computational grid boundary (meters)}
\\ 
IDATA \> =  0.0 \> \parbox[t]{\width}{ Regular run} \\ 
      \> =  1.0 \> \parbox[t]{\width}{ Reads from PBX data tape file ENINA} \\ 
      \> =  2.0 \> \parbox[t]{\width}{ Reads from TFTR data tape file ENINA} \\ 
      \> =  3.0 \> \parbox[t]{\width}{ Reads from D-III-D data tape file ENINA} \\ 
      \> =  4.0 \> \parbox[t]{\width}{ Reads from PBX-M data tape file ENINA} \\ 
      \> =  5.0 \> \parbox[t]{\width}{ Reads from PBX-M data tape file ENINA (New format)} \\
      \>  =  6.0 \> \parbox[t]{\width}{FEDTSC}\\ 
      \> =  7.0 \> \parbox[t]{\width}{TCV - Hofmann control algorithm \index{control!Hofmann algorithm}}\\
      \> = 8.0 \> \parbox[t]{\width}{ASDEX-U} \\
      \> =  9.0 \> \parbox[t]{\width}{ Reads from NSTX netCDF file nstx\_magnetics.nc} \\
      \> =  10. \> \parbox[t]{\width}{ Reads from MAST file mast\_magnetics.nc} \\
\end{tabbing} 
{\bf Note} : A problem with ISYM=0 and NZ=NZ0 will have the same zone size as a problem
with
ISYM=1 and NZ=(NZ0-1)/2 +1.
\newpage \subsubsection{Card 02 - Time step and switches} 
\index{time!step} 
\begin{tabbing} 
XXXXXX \= XXXXXX \= XXXXXX \= XXXXXX \= XXXXXX \= XXXXXX \=
XXXXXX \kill   
11 \> 21 \> 31 \> 41 \> 51 \> 61 \> 71 \\ 
\footnotesize DTMINS \>\footnotesize DTMAXS \>\footnotesize DTFAC \>\footnotesize
LRSWTCH \>\footnotesize IDENS \>\footnotesize IPRES \>\footnotesize IFUNC 
\end{tabbing}
\setw{LRSWTCH = 0.0X } 
\begin{tabbing} 
LRSWTCH \= = 0.0X \= \parbox[t]{\width}{ Normal run } \kill 
DTMINS \>       \> \parbox[t]{\width}{Minimum time step allowed ($\mu$sec).  Initial time step
is 2*DTMINS (0.001)} \\ 
DTMAXS \>       \> \parbox[t]{\width}{ Maximum time step allowed ($\mu$sec) (1.0)} \\ 
DTFAC  \>       \> \parbox[t]{\width}{ Time step safety factor. The time step used is the
maximum value that is theoretically stable multiplied by DTFAC with the
additional constraint that $\delta t$ increase by at most 20\% every 10 cycles and be less than
DTMAXS (0.5)} \\ 
LRSWTCH \> = 0.0 \> \parbox[t]{\width}{ Normal run} \\ 
                 \> = N.0 \> \parbox[t]{\width}{ Special test run\index{test run} for coils without plasma where coil
currents in group N are initialized to :\\
 \\
CCOILS(N)=SGN[ZCOIL(N)]*ACOEF(12)/RSWIRES(N) \\
 \\
To use this option TEVV must be positive, turn off all feedback and set IRST=0.0} \\ 
 \\
IDENS \>  = 0.0 \> \parbox[t]{\width}{ Regular calculation of density transport \index{transport!particle}
Particle diffusion and pinch coefficients are input via acoef(851) and (852).
Particle source terms provided by Neutral Beams [cards 25 and 23] and the
source function acoef(871-876). The boundary
conditions are provided by acoef(881) and the type 24 cards.
} \\ 
        \> = 1.0 \> \parbox[t]{\width}{ Forces the density to have profile given by} 
\end{tabbing} 
\begin{equation}
\label{eq:dens} 
R( \Psi ,t) = {\rm UDSD*RNORM(t)*}  \left\{     \left [ 1- \left ( \frac{\Psi-\Psi_{min}}{\Psi_{lim}
-\Psi_{min}} \right ) ^{\rm BETAR} \right ] ^{\rm ALPHAR} + r_{edge} \right\}
\end{equation} 
\begin{tabbing}
\label{eq:pres} 
LRSWTCH \= = 0.0X \= \parbox[t]{\width}{ Normal run } \kill 
      \>       \> \parbox[t]{\width}{ where ALPHAR and BETAR are input on type 04 or type
47,48 cards. The edge density $r_{edge}$ is input via acoef(881)} \\ 
IPRES \> = 0.0 \> \parbox[t]{\width}{ Regular calculation of energy transport}\index{transport!energy} \index{plasma!pressure}\\ 
           \> = 1.0 \> \parbox[t]{\width}{ Forces the pressure to equal one of the following
analytical forms :} \\ 
      \>       \> \\ 
      \>       \> For IFUNC=1,3,4,5 : 
\end{tabbing}  
\begin{eqnarray} 
P(\Psi,t) & = & P0(t) \left [ \frac{\Psi_{lim} -\Psi}{\Psi_{lim}-\Psi_{min}} \right ] ^{\rm ALPHAP} \\ 
 &+& {\rm ACOEF(110)*\left( \frac{ALPHAP}{ALPHAP+1} \right)}*\left[ \frac{\Psi_{lim}-\Psi}{\Psi_{lim}-\Psi_{min}} \right]^{\rm ALPHA +1} 
\end{eqnarray} 
\begin{tabbing} 
LRSWTCH \= = 0.0X \= \parbox[t]{\width}{ Normal run } \kill 
\>       \> For IFUNC=2 
\end{tabbing} 
\begin{equation} 
\frac {dP}{d\Psi} = P0 \left [ \frac {e^{-(\rm ALPHAP)\hat{\Psi}} - e^{-(\rm
ALPHAP)}}{e^{-(\rm
ALPHAP)} -1} \right ] . 
\end{equation} 
\begin{tabbing} 
LRSWTCH \= = 0.0X \= \parbox[t]{\width}{ Normal run } \kill 
\>       \> \parbox[t]{\width}{ where $\hat{\Psi} = (\Psi - \Psi_{min})/(\Psi_{lim}-\Psi_{min})$
and ALPHAP is input on type 13 card.  P0(t) is determined by card type 17. The ratio of electron to
ion pressure is given by ACOEF(2).  Pressure equilibration time is given by
EQRATE,ACOEF(4). } \\ 
IFUNC \>      \> \parbox[t]{\width}{ Switch to choose the functional forms to use for pressure
and toroidal field functions as described above and on type 13 card.} \\ 
      \> = 1.0 \> \parbox[t]{\width}{ tokamak profiles (Princeton)} \\ 
      \> = 2.0 \> \parbox[t]{\width}{ tokamak profiles (ORNL)} \\ 
      \> = 3.0 \> \parbox[t]{\width}{ RFP profiles (LANL)} \\ 
      \> = 4.0 \> \parbox[t]{\width}{ spheromak profiles (Princeton)} \\ 
      \> = 5.0 \> \parbox[t]{\width}{ ``ohmic'' profiles stationary on
      resistive time scale}\\
     \> = 6.0 \> \parbox[t]{\width}{Calls special subroutine {\bf splinfit}} \\
     \> = 7.0 \> \parbox[t]{\width}{prescribed $<J . B>$ profile}
\end{tabbing}
\newpage \subsubsection{Card 03 - Numerical} 
\begin{tabbing} 
XXXXXX \= XXXXXX \= XXXXXX \= XXXXXX \= XXXXXX \= XXXXXX \=
XXXXXX \kill   
11 \> 21 \> 31 \> 41 \> 51 \> 61 \> 71 \\ 
\footnotesize XLIM \>\footnotesize ZLIM \>\footnotesize XLIM2 \>\footnotesize FFAC
\>\footnotesize NDIV \>\footnotesize ICIRC \>\footnotesize ISVD 
\end{tabbing}
\setw{XLIM2 = 0.0X } 
\begin{tabbing} 
XLIM2 \= = 0.0X \= \parbox[t]{\width}{this is junk to set tabs} \kill 
XLIM \> \> \parbox[t]{\width}{An internal boundary is defined inside the computational grid so 
that the plasma can only occupy the rectangular region defined by : $ \rm XLIM < X < XLIM2
$ ,and $\mid \rm Z \mid < \rm ZLIM$. The region outside this is always treated as vacuum. (i.e. ,\
no plasma can exist there)}\\ 
ZLIM  \>  \>See XLIM\\ 
XLIM2 \>  \>See XLIM\\ 
FFAC \> \> \parbox[t]{\width}{Factor by which Alfv\'en waves are artificially slowed  \index{mass enhancement}
down.  Ion mass is increased by $\rm FFAC^2$. The time steps for the fast wave and 
the Alfv\'en wave are proportional to this.  If negative, uses the absolute value 
to initialize and adjusts FFAC according to ACOEF (801)-(805) to keep AMACH constant.
(1.0)}\\ 
NDIV \> \> \parbox[t]{\width}{Number of sub-cycles in diffusion part of poloidal flux \index{subcycling}
equation and in the fast wave equation. The time step for fast wave and resistive diffusion 
are inversely proportional to this, as discussed in section~(\ref{sec:1dvar}) (1.0)}\\ 
ICIRC \> = 0.0 \> \parbox[t]{\width}{Don't solve circuit equations\index{circuit equations} for the external coils}\\ 
      \> = 1.0 \> \parbox[t]{\width}{Do solve circuit equations for the external coils}\\ 
ISVD  \> = 0.0 \> \parbox[t]{\width}{Don't perform SVD analysis to obtain x-point}\\ 
      \> = 1.0 \> \parbox[t]{\width}{Do perform SVD analysis to obtain x-point} 
\end{tabbing}
\newpage \subsubsection{Card 04 - Surface Averaging} 
\index{surface averaging}
\begin{tabbing} 
XXXXXX \= XXXXXX \= XXXXXX \= XXXXXX \= XXXXXX \= XXXXXX \=
XXXXXX \kill   
11 \> 21 \> 31 \> 41 \> 51 \> 61 \> 71 \\ 
\footnotesize ISURF \>\footnotesize NPSI \>\footnotesize NSKIPSF \>\footnotesize TFMULT
\>\footnotesize  ALPHAR \>\footnotesize BETAR \>\footnotesize ITRMOD 
\end{tabbing}
\setw{NSKIPSF = 0.0X } 
\begin{tabbing} 
NSKIPSF \= = 0.0X \= \parbox[t]{\width}{this is junk to set tabs} \kill 
ISURF   \> = 0.0 \>No surface averaging (default)\\ 
        \> = 1.0 \>Use surface averaged transport equations\\ 
NPSI  \>   \> \parbox[t]{\width}{Number of PSI surfaces ($\Phi$ surfaces) for one-dimensional transport 
calculation (always must be $\rm NPSIT < NPSI$)}\\ 
NSKIPSF \> \> \parbox[t]{\width}{Number of cycles skipped between each surface average
calculation (20.)}\\ 
TFMULT \> \> \parbox[t]{\width}{Multiplier defining the toroidal flux domain used in surface
average calculation.  The initial NPSIT will be NPSI/TFMULT.  NPSIT will increase as the plasma grows,
but if NPSIT ever exceeds NPSI the program will terminate. }\\ 
ALPHAR \> \> \parbox[t]{\width}{Exponent for the prescribed density function\index{density profile} \index{plasma!density}(see type 02 card)
Will be overwritten if type 47 card is included. (0.5)}\\ 
BETAR \>  \> \parbox[t]{\width}{Exponent for the prescribed density function (see type 02
card). 
Will be overwritten if type 48 card is included. (2.0)}\\ 
ITRMOD \>  \> \parbox[t]{\width}{Switch selecting transport model}\\ 
\> = 1.0 \> \parbox[t]{\width}{The neo-ALCATOR model, where \index{transport model!neo-Alcator}\\  
 \\ 
$\chi_e$ = ACOEF(35) $\times 10.^{19} /n_e$ (mks)\\ 
$\chi_i$ = ACOEF(37) $\times 10.^{19}/n_e$ (mks)\\ 
d = ACOEF(39)\\ 
\\ 
If ACOEF(107) is greater than zero, these coefficients are enhanced according to the
Kaye-Goldston formula\\ 
 \\ 
ACOEF(35) = $\sqrt{{\rm ACOEF(35)^2 + CHIAUXS}}$\\  
 \\ 
where\\ 
 \\ 
CHIAUXS = [(ACOEF(107)*NE(0)*300.)/$\rm I_p ]^2$*PTOT\\ 
 \\ 
where $\rm I_p$ is the plasma current in amps and PTOT is the total power in watts.}\\
 \\ 
\> = 2.0 \> \parbox[t]{\width}{Coppi/Tang transport model\index{transport  model!Coppi/Tang}}\\ 
 \\ 
\> \> Need to input the following on type 11 cards :\\
 \\ 
\> \> ACOEF(121) - .08 ... Auxiliary heated transport coefficient\\
\> \> ACOEF(122) - .42  ... Ohmic heated transport coefficient \\
\> \> ACOEF(123) - 0.5  ... Constant in form factor\\
\> \> ACOEF(126) - 2.0  ... Ratio of $\chi_i$ to $\chi_e$  \\
\> \> \parbox[t]{\width} {For H-mode modeling, need to define 
      ACOEF(3003) and ACOEF(3011).  For ITB modeling, need to define
      ACOEF(3004)-ACOEF(3006).  Also, if idens=0, need to define ACOEF(851) 
      and ACOEF(875). }
\\

\> = 3.0 \> \parbox[t]{\width}{Coppi/Mazzucato/Gruber transport model (L.Sugiyama)} \\   
\> = 4.0 \> \parbox[t]{\width}{Coppi/Tang transport model (Englade)} \\ 
\> = 5.0 \> \parbox[t]{\width}{Marion Turner transport model} \\ 
\> = 6.0 \> \parbox[t]{\width}{GLF23 + neoclassical} \\ 
\> = 7.0 \> \parbox[t]{\width}{GLF23 + Coppi/Tang transport model} \\ 
\> = 8.0 \> \parbox[t]{\width}{GLF23 H-mode + Coppi/Tang+ neoclassical (see note below)} \\ 
\> = 9.0 \> \parbox[t]{\width}{GLF23 H-mode + neo-Alcator + neoclassical (see note below)} \\ 
\> = 10. \> \parbox[t]{\width}{MMM95 H-mode + Coppi/Tang + neoclassical (see note below)} \\ 
\> = 11. \> \parbox[t]{\width}{MMM95 H-mode + Neo-Alcator + neoclassical (see note below)} \\ 
\> = 12. \> \parbox[t]{\width}{calls special Kcoppi routine (C.Kessel)} \\ 
\\
\parbox[t]{\width}{Note that models 8.0-11.0 are H-mode models.  ACOEF(3011)
 needs to be defined as the H-mode pedistal location.   These models 
default to model 2.0 if there is no auxialliary 
heating.  Also, for models 9. and 11., ACOEF(35)-(39) and ACOEF(73) 
must be defined} \\

\end{tabbing}
\newpage \subsubsection{Card 05 - Limiter Points} 
\index{limiter}
\begin{tabbing} 
XXXXXX \= XXXXXX \= XXXXXX \= XXXXXXX \= XXXXXXX \= XXXXXXX \=
XXXXXX \kill   
11 \> 21 \> 31 \> 41 \> 51 \> 61 \> 71 \\ 
\footnotesize I \>\footnotesize XLIMA(I) \>\footnotesize ZLIMA(I) \>\footnotesize XLIMA(I+1)
\>\footnotesize ZLIMA(I+1) \>\footnotesize XLIMA(I+2) \>\footnotesize ZLIMA(I+2) 
\end{tabbing}
\setw{XLIMA(I) xx} 
\begin{tabbing} 
XLIMA(I) \= xx\= \parbox[t]{\width}{This is junk to set tabs} \kill 
XLIMA(I) \> \> \parbox[t]{\width}{The $x$ coordinate of limiter I.  If XLIMA(I)=0 for any I,
that
field and the rest of the cards are ignored.}\\ 
ZLIMA(I) \> \> \parbox[t]{\width}{The $z$ coordinate of limiter I.}\\ 
 \\ 
\> \> \parbox[t]{\width}{Up to 3 limiter points can be defined on each type 5 card.  The
minimum
value 
 of the poloidal flux amongst all limiter points, PSILIM, defines the plasma 
 boundary PSI=PSILIM} 
\end{tabbing} 
{\bf Note} : If ISYM=1 and ZLIMA(I)$<$0 for some I, this limiter point will be automatically
discarded and the remaining points will be renumbered to be consecutive.
\newpage \subsubsection{Card 06 - Divertor} 
\index{divertor}
\begin{tabbing} 
XXXXXX \= XXXXXX \= XXXXXX \= XXXXXX \= XXXXXX \= XXXXXX \=
XXXXXX \kill   
11 \> 21 \> 31 \> 41 \> 51 \> 61 \> 71 \\ 
\footnotesize IDIV \>\footnotesize PSIRAT \>\footnotesize X1SEP \>\footnotesize X2SEP
\>\footnotesize Z1SEP \>\footnotesize Z2SEP \>\footnotesize NSEPMAX 
\end{tabbing}
\setw{PSIRAT = 1.0X } 
\begin{tabbing} 
PSIRAT \= = 1.0X \= \parbox[t]{\width}{This is junk to set tabs} \kill 
IDIV \> = 0.0 \> \parbox[t]{\width}{Code does not check for magnetic divertor}\\ 
     \> = 1.0 \> \parbox[t]{\width}{Code will attempt to locate magnetic separatrix\index{separatrix} and use this
as the limiter if the value of PSI at the separatrix is less than PSILIM from the limiter points.}\\ 
PSIRAT \> \>\parbox[t]{\width}{Actual value of PSI used to limit plasma from separatrix is\\ 
 \\ 
PSILIM = PSIRAT*(PSISEP-PSIMIN)+PSIMIN\\ 
 \\ 
where PSISEP is the actual poloidal flux at the separatrix and PSIMIN is the flux at the magnetic
axis.  The normal value is PSIRAT = 0.999. (However, may have to lower if writing EQDSKA
files for JSOLVER)}\\ 
X1SEP \> \> \parbox[t]{\width}{The separatrix is only searched for in the region :\\
X1SEP$<$X2SEP}\\ 
X2SEP \> \> See X1SEP\\ 
Z1SEP \> \> \parbox[t]{\width}{The separatrix is only searched for in the region :\\
Z1SEP$<$ Z2SEP}\\ 
Z2SEP \> \> See Z1SEP\\ 
NSEPMAX \> \> \parbox[t]{\width}{The maximum number of separatrices that will be searched
for. (2.)}\\ 
\end{tabbing}
\newpage \subsubsection{Card 07 - Impurities} 
\index{impurities}
\begin{tabbing} 
XXXXXX \= XXXXXX \= XXXXXX \= XXXXXX \= XXXXXX \= XXXXXX \=
XXXXXX \kill   
11 \> 21 \> 31 \> 41 \> 51 \> 61 \> 71 \\ 
\footnotesize IIMP \>\footnotesize ILTE \>\footnotesize IMPBND \>\footnotesize IMPPEL
\>\footnotesize AMGAS \>\footnotesize ZGAS \>\footnotesize NTHE 
\end{tabbing}

\begin{tabbing} 
FRACOX \= = 0.0X \= \parbox[t]{\width}{this is junk} \kill 
IIMP \> = 0.0 \> No impurity line radiation.  ZEFF determined from card 12 (or 36) \\ 
     \> = 1.0 \> Impurity constant fraction of background density \\ 
     \>       \> ... see acoef(854)-(861) and also Type 82 (for time dependence) \\
     \> = 2.0 \> Impurity transport \\
     \>       \> for iimp .gt. 0, ZEFF determined from impurity concentration \\
ILTE \> = 0.0 \> Local thermodynamic equilibrium assumed\\ 
     \> = 1.0 \> Local thermodyanmic equilibrium not assumed\\ 
IMPBND \> = 0.0 \> non-flow boundary condition for impurities\\
     \> = 1.0 \> pedistal boundary condition for impurities\\
IMPPEL \> = 0.0 \> impurity pellet is deuterium\\
     \> = 1.0 \> impurity pellet is oxygen(not available)\\
     \> = 2.0 \> impurity pellet is carbon(not available)\\
     \> = 3.0 \> impurity pellet is iron(not available)\\
     \> = 4.0 \> impurity pellet is beryllium(not available)\\
     \> = 5.0 \> impurity pellet is neon\\
     \> = 6.0 \> impurity pellet is krypton\\
     \> = 7.0 \> impurity pellet is argon\\
AMGAS \>  \>  \parbox[t]{\width}{Mass of primary ion species in amu (1.0 for hydrogen, 2.0
for deuterium,
etc) (1.0)}\\ 
ZGAS \> \>  \parbox[t]{\width}{Charge of primary ion species (1.0 for hydrogen, 2.0 for helium,
etc) (1.0)}\\ 
NTHE \> \> \parbox[t]{\width}{Number of theta zones used in contouring plasma for parallel
impurity diffusion, and also in ballooning mode calculation (100.)} 
\end{tabbing} 
{\bf Note} : For IIMP$>$0, at least one of the fractions acoef(854)-(861) must be greater
than
zero OR these fractions must be defined as a function of time on the type 82 card.  
For IIMP=2, these values are used to initialize the impurity transport calculation. 
\newpage \subsubsection{Card 08 - Observation Pairs} 
\index{observation pairs}
\index{feedback!observation points}
\begin{tabbing} 
XXXXXX \= XXXXXX \= XXXXXX \= XXXXXX \= XXXXXX \= XXXXXX \=
XXXXXX \kill   
11 \> 21 \> 31 \> 41 \> 51 \> 61 \> 71 \\ 
\footnotesize J \>\footnotesize XOBS(2J-1) \>\footnotesize ZOBS(2J-1) \>\footnotesize XOBS(2J)
\>\footnotesize ZOBS(2J) \>\footnotesize NPLOTOBS 
\end{tabbing}
\setw{NPLOTOBS = 0.0X } 
\begin{tabbing}
NPLOTOBS \= = 0.0X \= Don't plot time histories \kill
J \> \> \parbox[t]{\width}{Number of observation pairs where poloidal flux difference is to be
recorded (and plotted optionally), and possibly used for feedback control (type 19 cards)}\\
XOBS(2J-1) \> \> $x$ coordinate of first point in pair\\
ZOBS(2J-1) \> \>$z$ coordinate of first point in pair\\
XOBS(2J) \> \>$x$ coordinate of second point in pair\\
ZOBS(2J) \> \>$z$ coordinate of second point in pair\\
NPLOTOBS \> = 0.0 \> \parbox[t]{\width}{Don't plot time histories}\\
         \> = 1.0 \> \parbox[t]{\width}{Do plot time history of fluxes and flux difference at
observation pair}\\
\\
\end{tabbing}
{\bf Note} : If ISYM=1 and ZOBS(J)$<$0 for some  J, this observation point will be
automatically discarded and the remaining observation points will be renumbered to be
consecutive. The values of NFEEDO on the type 19 cards will automatically be changed also to
reflect this renumbering.
\pagebreak
\newpage \subsubsection{Card 09 - External Coils}
\index{coils!external|(} 
\begin{tabbing} 
XXXXXX \= XXXXXX \= XXXXXX \= XXXXXXXX \= XXXXXXXX \=
XXXXXXXX \= XXXXXX \kill   
11 \> 21 \> 31 \> 41 \> 51 \> 61 \> 71 \\ 
\footnotesize N \>\footnotesize XCOIL(N) \>\footnotesize ZCOIL(N) \>\footnotesize
IGROUPC(N) \>\footnotesize ATURNSC(N) \>\footnotesize RSCOILS(N) \>\footnotesize
AINDC(N) 
\end{tabbing} 
Each type 09 card defines the properties of a single coil {\em external} to the computational grid.
\setw{ATURNSC(N) X}
\begin{tabbing} 
ATURNSC(N) X\= \parbox[t]{\width}{This is junk} \kill 
N \> \parbox[t]{\width}{External coil number (this must be a unique identifying number between
1 and PNCOIL)}\\  
XCOIL(N) \> \parbox[t]{\width}{The $x$ coordinate of the center of the external coil. (Must
be outside the computational grid)}\\ 
ZCOIL(N) \> \parbox[t]{\width}{The $z$ coordinate of the center of the external coil. (Must
lie outside the computational grid)}\\ 
IGROUPC(N) \> \parbox[t]{\width}{Group number of coil N.  Refers to type 15 card with the
same group number.} \index{coils!group number}\\ 
ATURNSC(N) \> \parbox[t]{\width}{Number of turns for coil N.  This is a positive or negative
number, not necessarily an integer.  The preprogrammed current for coil N will be the product
of ATURNSC(N) and the current in IGROUPC(N) as specified by the appropriate type 15 card}\\
RSCOILS(N) \> \parbox[t]{\width}{Resistance of coil N (ohms).  For multiturn coils, this is the
one-turn equivalent resistance.}\\ 
AINDC(N) \> \parbox[t]{\width}{Self inductance of coil N, assuming a single turn.  If a type 39
card is included, this will be overwritten with an inductance calculated from the geometry.} 
\end{tabbing} 
{\bf Note 1} : If ISYM=1 AND ZCOIL(N)$<$0 for some N, this coil will be automatically
discarded
and the remaining coils will be renumbered to be consecutive.\\ 
 \\ 
{\bf Note 2} : An external coil can belong to more than one coil group,
both for the feedback systems and for the preprogrammed currents.  To
specify the second group, follow the type 09 card with another type 09 card of the form:\\ 
 \\ 
09 10NN. ATURN(N,1) ATURN(N,2)...ATURN(N,6)\\ 
 \\ 
This will cause coil N to also belong to coil group NN with variable number of turns
ATURN(N,I) at time TPRO(I) as specified by the type 18 card.  Up to four additional systems
can be specified by using 10NN., 20NN.,30NN.,40NN., in the first field.\index{coils!external|)} 
\newpage \subsubsection{Card 10 - Internal Coils} 
\index{coils!internal|(}
\begin{tabbing} 
XXXXXX \= XXXXXX \= XXXXXX \= XXXXXXXX \= XXXXXXXX \=
XXXXXXXX \= XXXXXX \kill   
11 \> 21 \> 31 \> 41 \> 51 \> 61 \> 71 \\ 
\footnotesize M \>\footnotesize XWIRE(M) \>\footnotesize ZWIRE(M) \>\footnotesize
IGROUPW(M) \>\footnotesize ATURNSW(M) \>\footnotesize RSWIRES(M) \>\footnotesize
CWICS(M) 
\end{tabbing} 
Each type 10 card defines the properties of a single coil {\em internal} to the computational grid,
denoted a {\em wire}.\\ 
\setw{ ATURNSW(M) X}
\begin{tabbing} 
ATURNSW(M) X\= \parbox[t]{\width}{this is junk} \kill 
M \> \parbox[t]{\width}{Wire number (this must be a unique identifying number between 1 and
PNCOIL)}\\ 
XWIRE(M) \> \parbox[t]{\width}{The $x$ coordinate of the center of the wire. (coordinate
must lie inside the grid)}\\ 
ZWIRE(M) \> \parbox[t]{\width}{The $z$ coordinate of the center of the wire. (coordinate
must lie inside the grid)}\\ 
IGROUPW(M) \> \parbox[t]{\width}{The absolute value $\mid$IGROUPW(M)$\mid$ is the
group
number.  Refers to type 15 card with the same group number.\index{coils!group number}  If IGROUPW(M)$<$0, this is a
switch indicating that the wire is to occupy four adjacent cells (rather than 1) and to have the
relative number of turns in the four cell area weighted so that the current centroid will be at
[XWIRE(M),ZWIRE(M)].}\\ 
ATURNSW(M) \> \parbox[t]{\width}{The number of turns for wire M.  This is a positive or
negative number, not necessarily an integer.  The preprogrammed current for wire M will be the
product of ATURNSW(M) and the current in IGROUPW(M) as specified by the appropriate type
15 card.}\\ 
RSWIRES(M) \> \parbox[t]{\width}{The resistance of wire M (ohms).  If negative, resistance
is
major radius XWIRE(M) times the absolute value of RSWIRES(M).  For a multiturn coil, this
is
a one turn equivalent resistance.}\\ 
CWICS(M) \> \parbox[t]{\width}{Initial induced current in wire M. (kA)}\\ 
\end{tabbing}
{\bf Note 1} : If ISYM=1 AND ZWIRE(M)$<$0 for some M, this wire will be
automatically discarded and the remaining wires will be renumbered to be consecutive.\\ 
 \\ 
{\bf Note 2} : An internal coil can belong to more than one coil group for
feedback systems. To specify the second group, follow the type 10 card with another type 10 card
of the form:\\ 
 \\
10    10NN. ATURN(M,1) (M,2) (M,3) (M,4) (M,5) ATURN(M,6) \\
 \\
This will cause coil M to also belong to coil group NN with variable number of turns
ATURN(M,I) at time TPRO(I) as specified by type 18 card.  Up to four additional systems can
be specified by using 10NN.,20NN.,30NN.,40NN., in the first field.\\ 
 \\ 
{\bf Note 3} : If IGROUPW(M)$<$0, three new coils will be generated and the
parameter PNCOIL must be large enough to accommodate these.\\ 
{\bf Note 4} : At the end of a run, a special file is written out called NEWTYPE10.
This has CWIS defined.\\
\begin{tabbing}
{\bf Note 5} : \= Resistivity of copper :  1.724$\times 10^{-8} \Omega \cdot \rm m$\\
               \>Resistivity of Aluminum :  2.824$\times 10^{-8} \Omega \cdot \rm m$\\
               \>Resistivity of 304 SS :  7.2  $\times 10^{-7} \Omega \cdot \rm m$
\end{tabbing}\index{coils!internal|)}
\newpage \subsubsection{Card 11 - ACOEF Array}
\index{ACOEF array|(}
\begin{tabbing} 
XXXXX \= XXXXX \= XXXXXXX \= XXXXXXX \= XXXXXXX \= XXXXXXX \=
XXXXXX \kill   
11 \> 21 \> 31 \> 41 \> 51 \> 61 \> 71 \\ 
\footnotesize ICO \>\footnotesize NCO \>\footnotesize ACOEF(ICO) \>\footnotesize
$\ldots$(ICO+1) \> \footnotesize $\ldots$(ICO+2) \> \footnotesize $\ldots$(ICO+3) \>\footnotesize
ACOEF(ICO+4) 
\end{tabbing} 
\setw{ACOEF(1) X}
\begin{tabbing}
ACOEF(1) X\= set tabs \kill
ICO \> \parbox[t]{\width}{First index of ACOEF array specified on this card}\\
NCO \> \parbox[t]{\width}{The number of elements on this card. (1$\leq$NCO$\leq$5)}\\
ACOEF(I) \>\parbox[t]{\width}{The value of ACOEF(I).  The following ACOEF array elements
are presently defined (default values in parentheses):}\\
ACOEF(1) \> \parbox[t]{\width}{If 1.0,special run for PBX, if 2.0, special run for NSTX, if 3.0
special run for ASDEX-U.   If 4.0 or 5.0, special for ITER(0.).  Note that acoef(1997) has a special meaning for acoef(1)=4.}\\
ACOEF(2) \> \parbox[t]{\width}{Ratio of initial electron to total pressure (0.5)} \\
(3) \> \parbox[t]{\width}{Time interval over which feedback systems are turned on. (0.0)}\\
(4) \> \parbox[t]{\width}{Relaxation factor for pressure when IPRES=1 (100).}\\
(5) \> \parbox[t]{\width}{If 1.0, time history plots start new at restart time. (0.0)}\\
(6) \> \parbox[t]{\width}{Parallel diffusion multiplier for ISURF=0, IPRES=0 (10.)}\\
(7) \> \parbox[t]{\width}{Mix between Dufort Frankel and FTCS in flux diffusion (0.5)}\\
(8) \> \parbox[t]{\width}{Implicit parameter for surface averaged time advancement (1.0)}\\
(9) \> \parbox[t]{\width}{Numerical viscosity coefficient (40.)} \index{viscosity}\\
(10) \> \parbox[t]{\width}{Ratio of incompressible to compressible viscosity (0.5)}\\
(11) \> \parbox[t]{\width}{Proportionality constant in plasma current feedback (0.5)}\\
(12) \> \parbox[t]{\width}{Initial voltage in wires for LRSWTCH$>$0 (1.0)}\\
(13) \> \parbox[t]{\width}{IFLUX switch for poloidal flux boundary condition (see
Eq.~(\ref{eq:Psib}).  The options are : 0.0 constant, 1.0 first order, 2.0 second order, 3.0 full
integral, 4.0 Von-Hagenow's virtual casing method. (default is 4.0)}\\
(14) \> \parbox[t]{\width}{Maximum scale for SURFVOLT plot (10.0)}\\
(15) \> \parbox[t]{\width}{Minimum OH loop voltage (-100.0)}\\
(16) \> \parbox[t]{\width}{Maximum OH loop voltage (100.0)}\\
(17) \> \parbox[t]{\width}{ICUBE switch for cubic time point interpolation.  IF ICUBE=0.0
linear interpolation is used and for ICUBE=1.0 cubic interpolation is used. (0.0)}\\
(18) \> \parbox[t]{\width}{Multiplier in front of PSIDOT on boundary (0.0)}\\
(19) \> \parbox[t]{\width}{Not used}\\
(20) \> \parbox[t]{\width}{Switch for UCOR (0.0)}\\
(21) \> \parbox[t]{\width}{Error criterion for AMACH (1.0)}\\
(22) \> \parbox[t]{\width}{Error criterion for EKIN (100.)}\\
(23) \> \parbox[t]{\width}{EPSIMIN$\ldots$convergence criterion on PSI for
equilibrium($10^{-7}$)}\\ 
(24) \> \parbox[t]{\width}{EZCURF$\ldots$convergence criterion on  Z for
equilibrium($10^{-6}$)}\\
(25) \> \parbox[t]{\width}{DELG$\ldots$equilibrium parameter used for IFUNC=3 (1.0)}\\
(26) \> \parbox[t]{\width}{GRPRFP$\ldots$equilibrium parameter used for IFUNC=3 (1.0)}\\
(27) \> \parbox[t]{\width}{BETAJ$\ldots$equilibrium parameter used for IFUNC=3 (1.0)}\\
(28) \> \parbox[t]{\width}{Bypass initial filament growth rate calculation if nonzero (0.0)}\\
(29) \> \parbox[t]{\width}{Time in seconds at which calculation stops (1000.)} \index{time!stopping}\\
(30) \> \parbox[t]{\width}{IWAYNE$\ldots$switch to write special disruption file and produce
voltage plots at flux loops (0.0)}\\
(31) \> \parbox[t]{\width}{TJPHI$\ldots$time when to start writing (0.0)}\\
(32) \> \parbox[t]{\width}{DTJPHI$\ldots$time increment for writing (0.0)}\\
(33) \> \parbox[t]{\width}{TMOVIE (0.0)} \\
(34) \> \parbox[t]{\width}{DTMOVIE (0.0)} \\
(35),(37),(39) \> \parbox[t]{\width}{Anomalous transport coefficients CHIE,CHII,D
(1.0,1.0,0.2)   note:  These are multiplied by $10^{19}/n$, so the effective diffusion coefficient 
is $D = ACOEF(39) \times 10^{19}/n$ etc.}\\
(36),(38) \> \parbox[t]{\width}{off-diagonal transport coefficients for  CHIE,CHII (0.0,0.0)}\\
(40) \> \parbox[t]{\width}{If 1.0, U not zeroed in vacuum (0.0)}\\
(41) \> \parbox[t]{\width}{RESGAP$\ldots$coefficient of resistivity for gap in conductors (0.5).
Set to 0.0 for default value which is a very high resistance. The effect of the gap is to constrain zero net current in coil groups with
IGROUP$<$0 on type 15 cards.  If nonzero, that value will be used as the
as the resistance across the gap.  This value is overidden if type 33 card is  supplied.}\\
(42) \> \parbox[t]{\width}{Minimum $x$ for profile plots (0.)}\\
(43) \> \parbox[t]{\width}{Maximum $x$ for profile plots (0.)}\\
(44) \> \parbox[t]{\width}{IRFP$\ldots$set to 1.0 for reversed field pinch (0.0)}\\
(45) \> \parbox[t]{\width}{Number of zones to search over for x-point (2.0)}\\
(46) \> \parbox[t]{\width}{Maximum for $\tau _e$ plot (sec) (2.0)}\\
(47) \> \parbox[t]{\width}{Maximum power for problem with burn control (used to regulate
heating) ($1. \times 10^{12}$)}\\
(48),(49) \> \parbox[t]{\width}{The number of contours drawn in plasma and vacuum (20.,20.)}\\
(50),(51) \> \parbox[t]{\width}{Relaxation factors for initial equilibrium calculation (0.5,0.5)}\\
(52) \> \parbox[t]{\width}{Vacuum vessel poloidal inductance (0.0)}\\
(53) \> \parbox[t]{\width}{Vacuum vessel poloidal resistance (0.0)}\\
(54) \> \parbox[t]{\width}{Current feedback coefficient for burn control (0.0)}\\
(55) \> \parbox[t]{\width}{Reflectivity coefficient for cyclotron radiation (0.9)}\\
(56) \> \parbox[t]{\width}{HYPER heating multiplier (0.0)}\\
(57) \> \parbox[t]{\width}{t-begin for HYPER (0.0)}\\
(58) \> \parbox[t]{\width}{t-end for HYPER ($1. \times 10^6$)}\\
(59) \> \parbox[t]{\width}{EPSHYP$\ldots$convergence criteria in HYPER ($1. \times
10^{-6}$)}\\
(60) \> \parbox[t]{\width}{NLOOPM$\ldots$maximum iterations in HYPER (4000.)}\\
(61) \> \parbox[t]{\width}{If nonzero, ZMAG time history plotted even for ISYM=1 (0.0)}\\
(62) \> \parbox[t]{\width}{Ratio of toroidal to compressible viscosity (1.0)}\\
(63) \> \parbox[t]{\width}{Affects LSAW for ISURF=0 (0.667)}\\
(64) \> \parbox[t]{\width}{Hyperresistivity coefficient (0.0)}\\
(65) \> \parbox[t]{\width}{Hyperresistivity fraction (0.1)}\\
(66) \> \parbox[t]{\width}{Hyperresistivity exponent (4.0)}\\
(67) \> \parbox[t]{\width}{Hyperresistivity iteration damping-factor (1.2)}\\
(68) \> \parbox[t]{\width}{Hyperresistivity iteration safety factor (1.0)}\\
(69) \> \parbox[t]{\width}{Crash-time for Porcelli Sawtooth Model (0.10)}\\
(70) \> \parbox[t]{\width}{Relaxation for resistivity when LRSWTCH$\ne$0
($1.\times10^{-4}$)}\\
(71) \> \parbox[t]{\width}{Maximum temperature for resistivity calculation ($1.\times10^6$)}\\
(72) \> \parbox[t]{\width}{Bypass writing input on plot file if ACOEF(72)$>$0 (0.0)}\\
(73) \> \parbox[t]{\width}{fraction of ion neoclassical to use for electrons}\\
(74) \> \parbox[t]{\width}{Special limiter adjustment switch (0.0)}\\
(75) \> \parbox[t]{\width}{Number of cycles coil resistivity is enhanced to let perturbation
in(0.0)}\\
(76) \> \parbox[t]{\width}{Switch for setting FBFAC(I1) to FBFAC(I2) to zero after equilibrium
calculation (0.0)}\\
(77) \> \parbox[t]{\width}{I1 - see ACOEF(76) (0.0)}\\
(78) \> \parbox[t]{\width}{I2 - see ACOEF(76) (0.0)}\\
(79) \> \parbox[t]{\width}{IGONE = 0 for normal run, =1 if no closed flux surfaces (0.0)} \\
(80) \> \parbox[t]{\width}{Group number of superimposed oscillation (0.0)}\\
(81) \> \parbox[t]{\width}{Amplitude of oscillation (kA) (0.0)}\\
(82) \> \parbox[t]{\width}{Period of oscillation (seconds) (0.0)}\\
(83) \> \parbox[t]{\width}{2nd group number (0.0)}\\
(84) \> \parbox[t]{\width}{2nd amplitude number (0.0)}\\
(85) \> \parbox[t]{\width}{3rd group number (0.0)}\\
(86) \> \parbox[t]{\width}{3rd amplitude number (0.0)}\\
(87) \> \parbox[t]{\width}{4th group number (0.0)}\\
(88) \> \parbox[t]{\width}{4th amplitude number (0.0)}\\
(90) \> \parbox[t]{\width}{Drag terms in equation of motion (0.2)}\\
(91) \> \parbox[t]{\width}{Drag terms in equation of motion (0.2)}\\
(92) \> \parbox[t]{\width}{Drag terms in equation of motion (0.2)}\\
(93) \> \parbox[t]{\width}{Confinement time for He-ash (1.0).  This value is overidden
if type 73 card is supplied.} \\
(95) \> \parbox[t]{\width}{TDISRUPT$\ldots$time at which disruption occurs and QSAW
changes ($1.\times10^6$)}\\
(96) \> \parbox[t]{\width}{QSAW2$\ldots$value of QSAW after disruption (2.0)}\\
(97) \> \parbox[t]{\width}{Fraction of flux in plasma that halo extends beyond - a halo width.
(overwritten if type 61 card is included) (0.0)}\\
(98) \> \parbox[t]{\width}{Temperature of halo in eV (overwritten if type 60 card is included) (1.0)}\\
(99) \> \parbox[t]{\width}{Switch used in halo temperature and resistivity
calculation.  If acoef(99) $>$ 0, linearly interpolates temperature in
halo region from thalo at psilim to tevv at psilim+phalo. }\\
(101) \> \parbox[t]{\width}{IDTEST$\ldots$see note below (0.0)}\\
(102) \> \parbox[t]{\width}{VTEST$\ldots$see note below (0.0)}\\ 
  \\
  \\ 
{\bf Note} : Program will terminate normally if :\\
\begin{tabular}{lcllll}
IDTEST =  & -1 &  AND & $I_p(a)$ & $<$ & VTEST\\
''&  1 &`` &``       & $>$ &`` \\
''& -2 &`` &$Z_{MA}$ & $<$ &``\\
''&  2 &`` &''       & $>$ &''\\
''& -3 &`` &$\dot{I}_p(A/S)$ & $<$ & ``\\
''&  3 &`` &''&  $>$ &``\\
``& -4 &`` & $X_{MA}(m)$ & $<$ &``\\
``&  4 &`` &'' & $>$  & ``\\
``& -5 &'' & $q_{95}$  & $<$ & ``\\
``&  5 & `` & `` & $>$ & ``\\
``& -6 & `` & $\kappa$ & $<$ & ``\\
``&  6 & `` & `` & $<$ & ``\\
`` &-7 & `` & $\delta$ & $<$ & ``\\
`` & 7 & `` & `` & $>$ & ``\\
`` &-8 & `` & $PTOT(MW)$ & $<$ & ``\\
`` & 8 & `` & `` & $>$ & ``\\
`` &-9 & `` & CURR(1)(kA) & $<$ & ``\\
`` & 9 & `` & `` & $>$ & ``
\end{tabular}
 \\
 \\
(103) \> \parbox[t]{\width}{Multiplier bootstrap current (1.0)}\\
(104) \> \parbox[t]{\width}{ITEMP: If ITEMP=1, the temperature in the external coils will be calculated as a function of time.  In this case, type 39 cards must be included to provide additional coil information. (0.0)}\\
(106) \> \parbox[t]{\width}{Multiplier of AJLH for lower hybrid (1.0)}\\
(107) \> \parbox[t]{\width}{Anomalous auxiliary heating transport coefficient (1.0)} \index{transport!anomalous}\\
(108) \> \parbox[t]{\width}{ not presently used}\\
(109) \> \parbox[t]{\width}{No trapped particles when ACOEF(109)=1.0 (0.0)} \index{trapped particle!switch}\\
(110) \> \parbox[t]{\width}{Coefficient in pressure function (0.0) (see type 02 card)}\\
(111) \> \parbox[t]{\width}{Feedback constant for plasma density when IDATA=1 (0.0)}\\
(112) \> \parbox[t]{\width}{Feedback constant for ZEFF when IDATA=1 (0.0)}\\
(113) \> \parbox[t]{\width}{relative fraction of tritium for alpha heating calculation (0.49)}\\
(114) \> \parbox[t]{\width}{Stored energy wanted for feedback on D-T mix(0.0)}\\
(115) \> \parbox[t]{\width}{Derivative gain for control of feedback on DT mix(0.0)}\\
(120) \> \parbox[t]{\width}{Fraction of ETA (LSAW) to use in sawtooth model for ISAW=1.
     Note: 0 gives maximum flattening, 1 gives no flattening.(0.5)}\\
(121) \> \parbox[t]{\width}{Auxiliary heated transport coefficient for ITR=2 (0.08)}\\
(122) \> \parbox[t]{\width}{Ohmic transport coefficient for ITR=2 (0.42)}\\
(123) \> \parbox[t]{\width}{Factor added to $q_{cylin}$ for ITR=2 (0.5)}\\
(124) \> \parbox[t]{\width}{$\chi$ enhancement inside q=1 surface (for isaw=1) (2.0)}\\
(125) \> \parbox[t]{\width}{Feedback constant for $\chi$ when IDATA=1 (0.0)}\\
(126) \> \parbox[t]{\width}{Ratio of $\chi_i$ to $\chi_e$ (2.0)  If
negative, then ACOEF(126) is $\chi_i$ in $m^2/sec$ (spatial constant)}\\
(127) \> \parbox[t]{\width}{Feedback coefficient for $\chi$ time derivative term (0.0)}\\
(128) \> \parbox[t]{\width}{Minimum value for FBCHI (IDATA=1) (0.5)}\\
(129) \> \parbox[t]{\width}{Maximum value for FBCHI (IDATA=1) (2.0)}\\
(130)-(298) \> \parbox[t]{\width}{Special coefficients for thyristor voltage source model}\\
\end{tabbing}
{\bf Special coefficients for current feedback:} \\
 \\
\label{p:vback}
\underline{Coils inside grid:} \\
\index{feedback!voltage}
\setw{ACOEF(296) }
\begin{tabbing}
ACOEF(290) X\= = 1.0x IGRX \= NSTART+7*NFBX \= $<$X \= JX \= $<$X \= \kill
ACOEF(290) \> = 1.0 Lausanne feedback model \\
           \> = 2.0 Standard PID-model \\
 \\
PID-model :\\
 \\
ACOEF(291) \> NSTART : first coefficient for feedback\\
ACOEF(292) \> NFB : total number of feedback systems\\
ACOEF(293) \> NWPRINT : print cycle (coil currents and voltages)\\
 \\
\parbox[t]{\textwidth}{Having specified NSTART and NFB, the subsequent coefficients have to
be specified according to:} \\
 \\
ACOEF(J) \> IGROUPW: \> NSTART       \> $<$  \> J  \> $<$ \> NSTART+NFB-1 \\
         \> VGAINP:  \> NSTART+NFB   \> $<$  \> J  \> $<$ \> NSTART+2*NFB-1\\
         \> VGAIND:  \> NSTART+2*NFB \> $<$  \> J  \> $<$ \> NSTART+3*NFB-1\\
         \> VGAINI:  \> NSTART+3*NFB \> $<$  \> J  \> $<$ \> NSTART+4*NFB-1\\
         \> TFBON:  \> NSTART+4*NFB \> $<$  \> J  \> $<$ \> NSTART+5*NFB-1\\
         \> TFBOFF:  \> NSTART+5*NFB \> $<$  \> J  \> $<$ \> NSTART+6*NFB-1\\
         \> VOLTMAX:  \> NSTART+6*NFB \> $<$  \> J  \> $<$ \> NSTART+7*NFB-1\\
         \> TRAMP:  \> NSTART+7*NFB \> $<$  \> J  \> $<$ \> NSTART+8*NFB-1\\
where:\\
\>  IGROUP : \> coil group for feedback \\
\> VGAINP : \> proportional feedback constants (V/A) \\
\> VGAIND : \> differential (Vs/A)\\
\>  VGAINI : \> integral (V/As)   [note: if negative, then used as a \\
\> \> switch to enforce series connection]  \\
\>  TFBON : \> time when feedback system is switched on (s)\\
\>  TFBOFF : \> time when feedback system is switched off (s)\\
\>  VOLTMAX :\> maximum voltage (V)(one turn equiv. voltage)\\
\>  TRAMP : \> ramp time (s)\\
 \\
\underline{Coils outside grid :}\\
\index{feedback!voltage}
Feedback on external coil group currents is applied, if :\\
 \\
\>\underline{ICIRC=1} and \underline{ACOEF(294)$>$129} \\
 \\
ACOEF(294) \> NSTART : first coefficient for feedback \\
ACOEF(295) \> NFB : total number of coil groups for feedback\\
ACOEF(296) \>= 0.0 \> \parbox[t]{\width}{Use inductance matrix in voltage feedback model :}\\
  \\
           \> $V_{FB}^j = \left[ \beta_p^k (\Delta I^k) + \beta_d^k \left( \frac{d \Delta I^k}{dt}
\right) + \beta_I^k \left( \int \Delta I^k dt \right) \right] \cdot M_{k,j}.$\\
 \\
           \>\parbox[t]{\width}{where $\beta_p^k$, $\beta_d^k$, $\beta_I^k$ are the proportional
and derivative and integral gains, $\Delta I$ is the difference between the actual current and the
desire current and $M_{k,j}$ is the mutual inductance matrix}\\
\\
           \>= 1.0 \> Standard PID-model :\\
 \\
          \> $V_{FB}^k = \left( \beta_p^k (\Delta I^k) + \beta_d^k \left( \frac{d \Delta I^k}{dt}
\right) + \beta_I^k \left( \int \Delta I^k dt \right) \right) \cdot \delta_{k,j},$\\
 \\
          \> where $\delta_{k,j}$ is the identity matrix.\\
 \\
           \>= 2.0 \> Special for ZT-H\\
           \>= 3.0 \> Use the following voltage feedback model:\\
 \\
          \> $V_{FB}^k = \left( R_j I_j + \beta_p^k (\Delta I^k) + \beta_d^k \left( \frac{d \Delta
I^k}{dt} \right) + \beta_I^k \left( \int \Delta I^k dt \right) \right) \cdot \delta_{k,j},$\\
 \\
           \>= 4.0 \> ASDEX upgrade  \\
           \>= 5.0 \> call missionc \\
           \>= 6.0 \> call volt \\
 \\
\label{p:vback2}
ACOEF(J) \>  IGROUPC : \> NSTART       \> $<$  \> J  \> $<$ \> NSTART+NFB-1 \\
         \>  GAINPEG :  \> NSTART+NFB   \> $<$  \> J  \> $<$ \> NSTART+2*NFB-1\\
         \>  GAINDEG :  \> NSTART+2*NFB \> $<$  \> J  \> $<$ \> NSTART+3*NFB-1\\
         \>  GAINIEG :  \> NSTART+3*NFB \> $<$  \> J  \> $<$ \> NSTART+4*NFB-1\\
         \>  TONEG :  \> NSTART+4*NFB \> $<$  \> J  \> $<$ \> NSTART+5*NFB-1\\
         \>  TOFFEG :  \> NSTART+5*NFB \> $<$  \> J  \> $<$ \> NSTART+6*NFB-1\\
         \>  VMINEG :  \> NSTART+6*NFB \> $<$  \> J  \> $<$ \> NSTART+7*NFB-1\\
         \>  VMAXEG :  \> NSTART+7*NFB \> $<$  \> J  \> $<$ \> NSTART+8*NFB-1\\
where:\\
\>  IGROUPC : \> external coil group for feedback\\
\>  GAINPEG : \> proportional feedback constants (V/A) (else:(mu/s))\\
\>  GAINDEG : \> differential (Vs/A) (else: (mu))\\
\>  GAINIEG : \> integral (V/As) (else: (mu/s**2)\\
\>  TONEG : \> time when feedback system is switched on (s)\\
\>  TOFFEG : \> time when feedback system is switched off (s)\\
\>  VMINEG : \> minimum voltage (kV)/turn\\
\>  VMAXED : \> maximum voltage (kV)/turn
\end{tabbing}
The following information is used by the shape control subroutine {\bf tcv1}, when IDATA=7 \index{control!Hofmann algorithm|(}
on card 01.  The default values for ACOEF(300) to ACOEF(560) are all equal to zero.
\setw{(303)X = 1.0X  }
\begin{tabbing}
(300)X  \= = 1.0X \= set tabs \kill
(300) \> \> \parbox[t]{\width}{If ACOEF(307)=0, the number of plasma current elements is equal
to 2*ACOEF(300)}\\
(301) \> \> \parbox[t]{\width}{If ACOEF(307)=0 AND ACOEF(319)=1, the plasma current
elements cover the rectangular area defined by : \\
 \\
 ACOEF(302) $< x <$ ACOEF(302)\\
 ACOEF(303) $< z <$ ACOEF(304) }\\
 \\
(302) \> \>See (301)\\
(303) \> \>See (301)\\
(304) \> \>See (301)\\
(305) \> \>Not used \\
(306) \> \>Not used \\
(307) \> = 0.0  \> fixed plasma current elements\\
       \>= 1.0  \> \parbox[t]{\width}{current elements are periodically adapted to the
preprogrammed plasma shape}\\
(308) \> \> \parbox[t]{\width}{Overall shape control gain, expressed as a time constant (in
seconds) for the action of the control algorithm}\\
(309) \> = 0.0 \> \parbox[t]{\width}{boundary points are specified in the ACOEF array}\\
      \> = 1.0  \>\parbox[t]{\width}{boundary points are generated analytically using the variables
RZERV, AZERV, EZERV, DZERV on type 42, 43, 44 and 45 cards}\\
(310) \> \> Not used \\
(311) \> \>Shape control proportional feedback gain \\
(312) \> \>Shape control derivative feedback gain \\
(313) \> \>Not used \\
(314) \> \>Not used \\
(315) \> \> \parbox[t]{\width}{Exit time in seconds for program diagnostics}\\
(316) \> \>\parbox[t]{\width}{Feedback coefficient for plasma current control by acting on flux
at reference point.  Only used when ACOEF(318)=2.}\\
(317) \> = 1.0  \> TCV\\
      \> = 2.0  \> BPX\\
      \> = 3.0  \> SSAT\\
      \> = N.0  \> \parbox[t]{\width}{User supplied subroutine containing the information
discussed in section~(\ref{sec:control2}).  This information is required for the Hofmann control
scheme.}\\
(318) \> = 0.0  \> \parbox[t]{\width}{Plasma current is feedback controlled by applying an OH
moment.  OH group currents are defined in ACOEF(401) through ACOEF(450).  Feedback gain
is ACOEF(332)}\\
      \> = 1.0  \> \parbox[t]{\width}{Plasma current is feedback controlled by acting on the
boundary flux.  The gain is ACOEF(329) and the weight is preprogrammed (see ACOEF(2093)
, ACOEF(2193) etc )}\\
      \> = 2.0  \> \parbox[t]{\width}{Plasma current is feedback controlled by acting on flux at
reference point.  Trajectory of reference point is preprogrammed (see
ACOEF(2091),ACOEF(2092),ACOEF(2191),ACOEF(2192), etc).  The gain is ACOEF(316) and
the weight is preprogrammed as under 1 above.}\\
      \> = 3.0  \> \parbox[t]{\width}{Plasma current is not feedback controlled, but the total
volt-sec at reference point (same reference point as under 2 above) is feedback controlled to
follow a given time evolution, as defined in ACOEF(2095), ACOEF(2195), etc.  Gain is
ACOEF(333), weight is preprogrammed as under 1 above.}\\
(319) \> \>\parbox[t]{\width}{ Ratio of maximum to minimum width of finite element matrix.}\\ 
(320) \> \> \parbox[t]{\width}{Vertical position control proportional feedback gain. If
ACOEF(320)$\ne$0 group currents for a radial field moment must be given in ACOEF(451),
ACOEF(452), etc.}\\
(321) \> \>\parbox[t]{\width}{Not used}\\
(322) \> \>\parbox[t]{\width}{Ellipticity \nolinebreak control proportional feedback gain.
\nolinebreak If ACOEF(322)$\ne$0, group currents producing a quadrupole moment must be
specified in ACOEF(501), ACOEF(502) etc.  This is not recommended!}\\
(323) \> \>\parbox[t]{\width}{Ellipticity control derivative feedback gain. If ACOEF(323)$\ne$0,
group currents producing a quadrupole moment must be specified in ACOEF(501), ACOEF(502)
etc.  This is not recommended!}\\
(324) \> \> \parbox[t]{\width}{Control cycle time (in seconds).  This is the time interval for
applying the control algorithm}\\
(325) \> = 0.0 \> \parbox[t]{\width}{For plasma current to be calculated from flux loops using
finite elements}\\
      \> = 1.0 \> \parbox[t]{\width}{For the TSC plasma current to be used in shape
subroutines.}\\
(326) \> \>Not used\\
(327) \> \>\parbox[t]{\width}{Number of control cycles between successive element changes}\\
(328) \> \>\parbox[t]{\width}{If ACOEF(328)=1, the shape evolution between two given shapes
can be modified by using the type 44 card.  In this case FRAC=EZERW (see subroutine
tcvshap).}\\
(329) \> \>\parbox[t]{\width}{Feedback gain for plasma current control by acting on the flux at
the plasma boundary.  Used when ACOEF(318)=1.}\\
(330) \> = 0.0 \> \parbox[t]{\width}{Measurements are taken from psi-matrix using the
subroutine {\bf grap}.}\\
      \> = 1.0 \>\parbox[t]{\width}{Measurements are computed using the subroutines {\bf gf}
and {\bf gradgf}.}\\
(331) \> \>\parbox[t]{\width}{Damping coefficient for control algorithm (normally = 0.5)}\\
(332) \> \>\parbox[t]{\width}{Feedback coefficient for plasma current control by applying an OH
current moment (see ACOEF(401)$\ldots$).  Only used when ACOEF(318)=0.}\\
(333) \> \>\parbox[t]{\width}{Feedback coefficient for volt-sec preprogramming.  Only used
when ACOEF(318)=3.}\\
(334) \> = 0.0 \> \parbox[t]{\width}{Vessel currents are equal to sum of the wire currents}\\
      \> = 1.0 \> \parbox[t]{\width}{Vessel currents are computed from time derivative of flux}\\
(340) \> \>\parbox[t]{\width}{Number of plasma shapes specified using the ACOEF array}\\
(341) \> \>\parbox[t]{\width}{If equal to 1, weight of the top boundary point is preprogrammed.}\\
(342) \> \>\parbox[t]{\width}{If equal to 1, weight of the bottom boundary point is
preprogrammed.}\\
(380) \> \>\parbox[t]{\width}{Number of preprogrammed boundary points (should be
approximately 2$\times$(Number of poloidal field coil groups))}\\
(401)-(450) \> \> \parbox[t]{\width}{Group currents which produce a perfect OH field.  Used
only when ACOEF(318)=0. Currents should be scaled such that the sum of all OH currents is
of the order 10 kA-turns}\\
(451)-(500) \> \> \parbox[t]{\width}{Group currents which produce a pure radial field.  Only
used when ACOEF(320)$\ne$0 or ACOEF(2096)$\ne$0.  Currents should be scaled as above}\\
(501)-(504) \> \> \parbox[t]{\width}{sed for Halo Current feedback simulations if acoef(501) nonzero}\\
(504) \> \> \parbox[t]{\width}{applied electric field for CHI} \\
(510) \> \> \parbox[t]{\width}{factor for determining the regularization parameter for shape feedback
(IPEXT=25 on type 20 card) with type 62,63 cards}\\
(511) \> \> \parbox[t]{\width}{Integral gain factor for shape feedback
(IPEXT=25 on type 20 card) with type 62,63 cards}\\
(512) \> \> \parbox[t]{\width}{Time for starting up the shape feedback slowly
(IPEXT=25 on type 20 card) with type 62,63 cards}\\
(513) \> \> \parbox[t]{\width}{Proportional gain factor for shape feedback
(IPEXT=25 on type 20 card) with type 62,63 cards}\\
(514) \> \> \parbox[t]{\width}{Derivative gain factor for shape feedback
(IPEXT=25 on type 20 card) with type 62,63 cards}\\
(540) \> \> \parbox[t]{\width}{Number of preprogrammed BSUBR=0 points}\\ 
(560) \> \>\parbox[t]{\width}{Number of preprogrammed BSUBZ=0 points} \index{control!Hofmann algorithm}\\
 \\
\parbox[t]{\textwidth}{{\bf Note} : Additional information for the Hofmann shape control
algorithm is provided on ACOEF(2000)-ACOEF(3000)}
\end{tabbing}
\setw{(808)X }
\begin{tabbing}
(888)X \= PHI2 (1.75) XXX FOR XXX \= ITYPE=1 XXX \= \kill
(700) \> NSLHRT : number of cycles skipped between ray tracing in LSC (50.)\\
(701) \> \parbox[t]{\width}{NSLHPC : number of cycles skipped between power and current calls to LSC
(10.). Note that NSLHPC $<$ NSLHRT}\\
 \\
\parbox[t]{\textwidth}{{\bf Note} : 700-705 also are used to define divertor plots} \\
\parbox[t]{\textwidth}{{\bf Note} : 701-705 used in special summary plot routine if idiv .ne. 0} \\
\parbox[t]{\textwidth}{These coefficients are needed for IFFAC=1 (neg FFAC on type 03) to
control automatic adjustment of FFAC.}\\
  \\  
(703) \> gcmin: possible min for group current plot, default 0. \\
(704) \> gcmax: possible max for group current plot, default 0. \\
(705) \> possible min for power plot, default 0. \\
(706) \> possible max for power plot, default 0. \\
  \\  
(710) \> nplot -- number of divertor blow-up plots (max 4) \\
(711) \> xmin(i),i=1,4 \\
(712) \> xmax(i),i=1,4 \\
(713) \> zmin(i),i=1,4 \\
(714) \> zmax(i),i=1,4 \\
  \\  
(751) \> xmin for second summary plot \\
(752) \> xmax for second summary plot \\
(753) \> zmin for second summary plot \\
(754) \> zmax for second summary plot \\
(755) \> multiplier of nx and nz for points used in contour plot \\
  \\  
(760) \> pellet run if 1.0 \\
(761) \> VXPEL...initial R velocity of pellet (note...normally negative) \\
(762) \> VZPEL...initial Z velocity of pellet \\
(763) \> XPEL....initial R position of pellet \\
(764) \> ZPEL....initial Z position of pellet \\
(765) \> RADPEL..initial radius of pellet (assumed spherical) \\
(766) \> time pellet gets injected \\
(767) \> fraction of impurity in pellet (set 0.0 for pure H pellet) \\
\\
(770) \> second pellet if 1.0 \\
(771) \> VXPEL...initial R velocity of second pellet (note...normally negative) \\
(772) \> VZPEL...initial Z velocity of second pellet \\
(773) \> XPEL....initial R position of second pellet \\
(774) \> ZPEL....initial Z position of second pellet \\
(775) \> RADPEL..initial radius of second pellet (assumed spherical) \\
(776) \> time second pellet gets injected \\
(777) \> fraction of impurity in second pellet (set 0.0 for pure H pellet) \\
  \\  
(778) \> time between subsequent pellets \\
(779) \> final time \\
\\
(790) \> = 0.0  \> original calculation of pellet density\\
      \> = 1.0  \> average pellet density source over trail\\
(791) \> = 0.0  \> original calculation of density integration for plots\\
      \> = 1.0  \> start density integration at restart time\\
\\
(792)  \>  fraction which volume extends for pellet backaveraging\\
(795)  \> = 1.0 for runaway calculation \\
(801)  \>  maximum AMACH (0.005)\\
(802)  \> minimum FFAC decrease (0.9)\\
(803)  \> maximum FFAC increase (1.1)\\
(804)  \> maximum FFAC (1000.)\\
(805)  \> minimum FFAC (1.0)\\  
(806)  \> Boundary relaxation factor (1.0)\\
 \\
(809)  \> Multiplier of pellet ablation rate (1.0)\\
(810)  \> multiplies $\eta$ (1.0)\\
(811)  \> QLIM : plasma will be limited by surface where q$\geq$QLIM (0.0)\\
(815) \> in missionc \\
(816) \> reserved for density jet \\
(817) \> reserved for density jet \\
(818) \> reserved for density jet \\
(819) \> reserved for density jet \\
(820) \> reserved for density jet \\
 \\
These coefficients are needed for subroutine ITERATE \\
 \\
(821) \> PHI2 (1.75) for \> ITYPE=1 \> poloidal flux \\
(822) \> SF (1.4)        \> `` \\
(823) \> FACCONV ($1.\times10^{-8}$) \> ``\\
(824) \> NIMAX (4000.) \> ``\\
 \\
(831) \> PHI2 (1.85) for \> ITYPE=2 \> velocity stream function\\
(832) \> SF (1.4) \> ``\\
(833) \> FACCONV ($1.\times10^{-8}$) \> `` \\
(834) \> NIMAX (2000.) \> ``\\
 \\
(841) \> PHI2 (1.62) for \> ITYPE=3 \> velocity potential\\
(842) \> SF (1.38) \> ``\\
(843) \> FACCONV ($1.\times10^{-8}$) \> `` \\
(844) \> NIMAX (4000.) \> ``\\
 \\
(850) \> Initial voltage for equilibrium calculation (0.0)\\
(851) \> \parbox[t]{\width}{Particle diffusion coefficient ($\rm m^2/s$) 
for IDENS=0 and ITRMOD=2.  See also ACOEF(875) (0.0)}\\
(852) \> \parbox[t]{\width}{Normalized pinch term (dimensionless).  This is the 
exponential decay factor for the steady state particle radial density profile (0.0)}\\
(853) \> Flux of impurities crossing outermost flux surface (\#/s) (0.0)\\
    \>         \\
NOTE \> Acoef(854)-(861) are for impurity fractions \\
     \> These can also be input as time-dependent variables on Type 82 \\
    \>         \\
(854) \> Oxygen (0.0)\\
(855) \> Carbon (0.0)\\
(856) \> Iron (0.0) \\
(857) \> Berillium (0.0)\\
(858) \> Neon (0.0)\\
(859) \> Krypton (0.0) \\
(860) \> Argon (0.0) \\
(861) \> Tungston (0.0) \\
(869) \> \parbox[t]{\width}{VT : transfer voltage(kV) for ZTH circuit when IRFP=1 and
ACOEF(296)=2; when $V_T \leq$ACOEF(860), OH power supply comes on. (0.0)} \\
(870) \> $\alpha$ for Ohm's law (0.0) : $\vec E + {\rm \vec{v}} \times \vec B = \eta \left( \vec 
J - \alpha \vec B / \mu_o \right)$ \index{Ohm's law}\\
(871) \> Edge density source multiplier in $atoms/m^3$  for IDENS=0, ISURF=1 \\
\> edge density source for IDENS=0., ACOEF(871-876) \\
\> $S=a_{871} \exp{a_{872} (\frac{\Phi - \Phi_{e}}{\Phi_{e}})}$ \\
\> $D_{\bot}=a_{851}+a_{a875} \tilde{\Phi}^{2}$ for $\tilde{\Phi} \leq 0.75$ \\
\> $D_{\bot}=a_{876}$ for $\tilde{\Phi} > 0.75$ \\
(872) \> Edge density source exponential decay factor for IDENS=0, ISURF=1\\
(875) \> Quadratic term multiplier in particle diffusion coefficient for IDENS=0, ISURF=1 \\
(876) \> Particle diffusion coefficient for $\phi>0.75$ for IDENS=0, ISURF=1\\
(877) \> Multiplier of PBREM (1.0) \\
(880) \> $T_{edge}$ (eV) for transport calculations. (0.0)\\
  \\
\>\parbox[t]{\width}{Note :  The electron  and  ion edge temperatures  are  deterined  as follows : \index{edge temperature} \index{temperature!edge} \label{p:tedge}}\\
 \\
 \>\parbox[t]{\width}{For ACOEF(880)=0.0  :  $\left({T_{edge}}\right)_e$ = TEVV  unless THALO  is  
specified  by ACOEF(98) or by the type 60  card, and WHALO is specified to be non-zero by ACOEF(97) 
or by the type 61.  Then $\left({T_{edge}}\right)_e$~=~THALO}  \\
  \\
  \>For ACOEF(880)$>$0.0 :  $\left({T_{edge}}\right)_e$ =ACOEF(880) \\
  \\
  \>In all cases  : $\left({T_{edge}}\right)_i$ = (ACOEF(882)-1.)$\left({T_{edge}}\right)_e$\\
 \\  
(881) \> Fraction of $n_o$ for edge density (0.1)\\
(882) \> Ratio of total pressure to electron pressure at edge (2.0)\\
(889) \> Ratio of n1 and n2 in density profile (acoef 3013 and 3014). It is now input on type 89 \\
(890) \> Heat conduction multiplier (1.0)\\
(891) \> Heat conduction denominator used in temperature equilibration (100.)\\
(894) \> ratio of pressure on sep to axis for FRC stability calculation \\
(895) \> Let x-point and psimin exist in structure \\
(896) \> Set to 1.0 for velocity chopping \\
(897) \> convective multiplier(3./2.) \\
(901) \> equilibrium shape control parameter  \\
       \> 1.0 \> only shape points are used  \\
       \> 2.0 \> shape points + flux linkage (acoef(902)) at xplas,zplas  \\
       \> 3.0 \> shape points, + x-point ($r_x$=acoef(903),$z_x$=acoef(904))  \\
       \> 4.0 \>  shape points + flux linkage + x-point   \\
(905) \>  \parbox[t]{\width}{ specifies max number of coil group currents to calculate
(actually set this equal to the total number of groups. To fix any
coil current, set the desired value in gcur(2) and set the corresponding
gcur(3) value to 1.0)} \\
(906) \> is the relative error tolorance...[1.e-3]  \\
(907) \> is the iteration number when shape feedback starts   \\
(908) \> is the iteration number when type 19 feedback ends  \\
(909) \> is the relaxation factor for equilibrium shape feedback  \\
(910) \> is number of iterations between resetting sigmax and relaxation factors \\
(911) \> is number of iterations to full implementation of vsec constraint \\
(931) \> switch for writing ographa file (0.0) \\
(941-962) \> used in routine jpolo for calculating  vessel forces for iwayne $>$ 0 \\
(950) \> special SPDD3D coding for DIII-D \\
(991)\>  \parbox[t]{\width} {ACOEF(991-997) are used with the $<j.B>$ current
profile equilibrium mode (IFUNC=7.0, on type 02 card)} \\
\>  \parbox[t]{\width} {$ \frac{\langle \vec{j} \cdot \vec{B} \rangle}{\langle
\vec{B} \cdot \nabla \phi \rangle} = a_{991} (1 - \hat{\psi}^{a_{992}})^{a_{993}}
+ (1-a_{991}) \frac{a_{994}^{2} \hat{\psi}^{a_{995}} (1-\hat{\psi})^{a_{996}}}{a_{994}^{2} +
(\hat{\psi}-a_{997})^{2}}$} \\
(1997) \> special switch for acoef(1)=4 to revert to 1997 code version treatment of iter VV \\
\end{tabbing}
The following is additional information for the Hofmann control scheme :\index{control!Hofmann algorithm|(}\\
\setw{(2081,2082,2083)X }
\begin{tabbing}
(2081,2082,2083)X \= set the tabs \kill
(2000) \> \parbox[t]{\width}{Time when the first plasma shape is specified (seconds)}\\
(2001-2030) \> \parbox[t]{\width}{The {\em x} coordinates of the boundary points}\\
(2031-2060) \> \parbox[t]{\width}{The {\em z} coordinates of boundary points}\\
(2061) \> Weight of top boundary point\\
(2062) \> Weight of bottom boundary point\\
(2071,2072,2073) \> \parbox[t]{\width}{{\em x, z}, weight of first BR=0 point}\\
(2074,2075,2076) \> \parbox[t]{\width}{{\em x, z}, weight of the second BR=0 point}\\
(2081,2082,2083) \> \parbox[t]{\width}{{\em x, z}, weight of first BZ=0 point}\\
(2084,2085,2086) \> \parbox[t]{\width}{{\em x, z}, weight of second BZ=0 point}\\
(2091,2092) \> \parbox[t]{\width}{{\em x, z} for preprogrammed volt seconds}\\
(2093) \> \parbox[t]{\width}{Weight of flux control, used when ACOEF(318)=1, 2, or 3.}\\
(2094) \> \parbox[t]{\width}{D-matrix scaling factor}\\
(2095) \> \parbox[t]{\width}{Preprogrammed volt seconds, used when ACOEF(318)=3.}\\
(2096) \> \parbox[t]{\width}{Derivative gain for ver{\nolinebreak}tical \nolinebreak position
control. If ACOEF(2096)$\ne$0, group currents for a radial field moment must be given in
ACOEF(451), ACOEF(452) etc.}\\
(2100)-(2195) \> Same as above for second plasma shape\\
(2200)-(2295) \> Same as above for third plasma shape\\
(etc) \> \parbox[t]{\width}{Continue in same fashion for all plasma shapes} \index{control!Hofmann algorithm|)}
\index{ACOEF array|)} \\
(3001) \> = 1.0 to write special UFILE data,=2.0 for 1d and 2D UFILES \\
(3003) \> transport multiplier for Coppi/Tang for flux $>$ acoef(3011) \\
       \> can be input as time-dependent on type 78 \\
(3004) \> first ITB coefficient:  minimum flux fraction \\
(3005) \> second ITB coefficient: maximum flux fraction \\
(3006) \> third ITB coefficient:  reduction multiplier \\
       \> can be input as time-dependent on type 80 \\
(3007) \> feedback coefficient for Chi Multiplier (0.0) \\
(3008) \> second feedback coefficient for Chi Multiplier (1.0) \\
(3009) \> spatial relaxation factor for GLF23 (.01) \\
(3010) \> time relaxation factor for GLF23 (0.1) \\
(3011) \> \parbox[t]{\width}{flux fraction for Coppi/Tang to apply acoef(3003) multiplier.  Also, H-mode pedistal location for transport models 8-10.
The GLF23 and MMM95 models are only applied interior to this fraction. (.75)    Can be input as time dependent
using the type 79.} \\
\end{tabbing} 
\setw{Note :}
{\bf Note} : \parbox[t]{\width}{In order to use the Hofmann control scheme, the additional
information described in section~(\ref{sec:control2}) must be provided through a subroutine.}\\
\setw{(acoef(3012))X }
\begin{tabbing}
(acoef(3012))X \= set the tabs \kill
(3012) \> \parbox[t]{\width}{flag for setting exponents n1/n2 in density profile if $>$0 (0)} \\
(3013) \> \parbox[t]{\width}{n1} \\
(3014) \> \parbox[t]{\width}{n2 } \\
(3015) \> \parbox[t]{\width}{exponential factor in $\chi$ at the edge used in Coppi model (0)} \\
(3101) \> \parbox[t]{\width}{hyper-conductivity coefficient to smooth temperature (0.008 to 0.020)} \\
(3102) \> \parbox[t]{\width}{tped (temperature pedestal) for use with hyper-conductivity (also on time dependent card type 81)}\\
(3208) \> \parbox[t]{\width}{switch to use EPED1 calculations: if $=$1 then TSC interpolates between hardwired values and adjust fhmodei. NOTE: the table contains EPED1 output for current of 7-10MA and 
                             pedestal density of $4-10\times10^{19}$. For density and current outside the tabulated range, the pedestal temeprature is not accurate.}
(4948) \> \parbox[t]{\width}{switch for using external diffusivity data if $>$0 (0)} \\
(4950) \> \parbox[t]{\width}{time to start TRANSP coupling ($10^8$)} \\
(4951) \> \parbox[t]{\width}{time step for TRANSP forward integration (0.0)} \\
(4952) \> \parbox[t]{\width}{} \\
(4953) \> \parbox[t]{\width}{\# of grids for profiles written to plasma state} \\
(4954) \> \parbox[t]{\width}{zimp} \\
(4955) \> \parbox[t]{\width}{H fraction in thermal ion species to be outputted to PS (0.0) (note: acoef(113)= D fraction)} \\ 
(4956) \> \parbox[t]{\width}{beam power for each NBI used by TRANSP/NUBEAM (0.0)} \\
(4957) \> \parbox[t]{\width}{number of Newton iterations for GLF linearization (1)} \\
(4958) \> \parbox[t]{\width}{step size for derivatives in Newton iteration for GLF (-1.0)} \\
(4959) \> \parbox[t]{\width}{switch for SWIM if $>$0 (0=no)} \\
(4960) \> \parbox[t]{\width}{switch for TRANSP coupling if $>$0 (0=no)} \\
(4961) \> \parbox[t]{\width}{switch for writing plasma state if $>$0 (0=no)} \\
(4962) \> \parbox[t]{\width}{switch for using TSC in beam heating/current drive (0=yes)} \\
(4963) \> \parbox[t]{\width}{switch for using TSC in fast ion source (0)} \\
(4964) \> \parbox[t]{\width}{switch for using TSC in fast wave heating/current drive (0)} \\
(4965) \> \parbox[t]{\width}{switch for using TSC in lower hybrid heating/current drive (0)} \\
(4966) \> \parbox[t]{\width}{switch for using TSC in ECH heating/current drive (0)} \\
(4967) \> \parbox[t]{\width}{switch for using external beam data if $>$0 (0)} \\
(4968) \> \parbox[t]{\width}{switch for using external lower hybrid data if $>$0 (0)} \\
(4969) \> \parbox[t]{\width}{switch for using external fast wave data if $>$0 (0)} \\
(4970) \> \parbox[t]{\width}{switch for using external electron density data if $>$0 (0)} \\
(4971) \> \parbox[t]{\width}{fraction of RF power to electrons when using analytic internal algorithm (1.0)} \\
(4972) \> \parbox[t]{\width}{switch for using external electron rotation data if $>$0 (0)} \\
(4973) \> \parbox[t]{\width}{switch to allow stored energy control using NB power if $>$0 (0)} \\
(4974) \> \parbox[t]{\width}{switch to allow stored energy control using RF power if $>$0 (0)} \\
(4975) \> \parbox[t]{\width}{switch for nb internal model, standard 0 (0)} \\
(4976) \> \parbox[t]{\width}{switch for icrf internal model, standard 0 (0)} \\
(4977) \> \parbox[t]{\width}{switch for density profile (0)} \\
(4978) \> \parbox[t]{\width}{switch for using external line radiated power if $>$0 (0)} \\
(4979) \> \parbox[t]{\width}{switch for using external ECRH data if $>$0 (0)} \\
(4980) \> \parbox[t]{\width}{stop if reading PS failed if 1 (0)} \\
(4981) \> \parbox[t]{\width}{itport(1) in GLF (0)} \\
(4982) \> \parbox[t]{\width}{itport(2) in GLF (1)} \\
(4983) \> \parbox[t]{\width}{itport(3) in GLF (1)} \\
(4984) \> \parbox[t]{\width}{itport(4) in GLF (0)} \\
(4985) \> \parbox[t]{\width}{itport(5) in GLF (0)} \\
(4986) \> \parbox[t]{\width}{angular velocity used in GLF (0.0)} \\
       \> \parbox[t]{\width}{angrotp(i)=acoef(4986)*(npsit-i)/npsit} \\
(4987) \> \parbox[t]{\width}{alpha value for the "alpha-effect" in GLF (0.0)} \\
(4988) \> \parbox[t]{\width}{switch for rotational stabilization in GLF (0=off)} \\
(4989) \> \parbox[t]{\width}{} \\
(4990) \> \parbox[t]{\width}{} \\
(4991) \> \parbox[t]{\width}{if .gt.0; get pressure profile from trxpl} \\
(4992) \> \parbox[t]{\width}{if .gt.0; get electron pressure profile from trxpl} \\
(4993) \> \parbox[t]{\width}{if .gt.0; get density profile from trxpl} \\
(4994) \> \parbox[t]{\width}{if .gt.0; get nubeam machine description and wall data from trxpl} \\
\end{tabbing}
\newpage \subsubsection{Card 12 - Transport} 
\begin{tabbing} 
XXXXXX \= XXXXXX \= XXXXXX \= XXXXXX \= XXXXXX \= XXXXXX \=
XXXXXX \kill   
11 \> 21 \> 31 \> 41 \> 51 \> 61 \> 71 \\ 
\footnotesize TEVV \>\footnotesize DCGS \>\footnotesize QSAW \>\footnotesize ZEFF
\>\footnotesize IALPHA \>\footnotesize IBALSW \>\footnotesize ISAW 
\end{tabbing} 
\setw{IBALSW = 0.0X }
\begin{tabbing} 
IBALSW \= = 0.0X \= \parbox[t]{\width}{This is junk to set tabs} \kill 
TEVV \> \> \parbox[t]{\width}{Temperature of the vacuum region for use in resistivity \index{vacuum temperature} \index{temperature!vacuum}
calculation.  If TEVV is negative, (-TEVV) is used initially then TEVV is adjusted to give the
maximum value which is numerically stable.  If type 34 card is included, this overrides value
specified here. (1.0)}\\
DCGS \> \> \parbox[t]{\width}{Reference number density in units of $10^{19} / \rm m^3$ \index{plasma!density} 
The
actual density for IDENS=1 is the product of DCGS and RNORM on type 24 card.}\\
QSAW \> \> \parbox[t]{\width}{The resistivity is enhanced in the center of the plasma if
ISURF=1 and ISAW=1 and the local safety factor satisfies q$<$QSAW.
(see acoef(120) and acoef(124)) (1.0)}\\
ZEFF \> \> \parbox[t]{\width}{The effective Z used in the resistivity calculation if iimp=0 on type 07. 
Can also be input as a time dependent function on type 36. (1.0)} \index{resistivity enhancement}\\
IALPHA \> \> \parbox[t]{\width}{Switch for $\alpha$-particle heating. If IALPHA=1, the \index{heating!fusion}
$\alpha$-particle heating corresponding to a 50:50 D/T mixture is included in the energy
equation. (0.0)}\\

IBALSW \> \> Switch for ballooning calculation (0.0) \index{ballooning calculation} \index{stability!ballooning calculation}\\
       \> = 0.0 \> No ballooning calculation\\
       \> = 1.0 \> \parbox[t]{\width}{Ballooning calculation performed every NSKIPSF cycles on
every flux surface.  Results are presented as a stability plot at the end of the calculation. {\em
WARNING :} may be expensive for time dependent calculations }\\
       \> = 2.0 \> \parbox[t]{\width}{Same as 1.0, with the addition that the thermal conductivity
is increased by a factor of 10 on all surfaces found to be unstable.}\\
  \> = 3.0 \> \parbox[t]{\width}{Same as 2.0, except only surfaces in the SOL are checked.}\\
ISAW \> \> Switch for sawtooth model (1.0) \\
       \> = 1.0 \> \parbox[t]{\width}{old "Standard" sawtooth model 
                 (average in time, based on qsaw)  Also, increases Chi and eta inside
                 q=qsaw surface based on ACOEF(120) and (124)}\\
       \> = 2.0 \> \parbox[t]{\width}{Kadomtsev Sawtooth at times specified on type 75 card}\\
       \> = 3.0 \> \parbox[t]{\width}{Full Porcilli Model}\\



\end{tabbing}
\newpage \subsubsection{Card 13 - Initial Conditions}
\begin{tabbing}
XXXXXX \= XXXXXX \= XXXXXX \= XXXXXX \= XXXXXX \= XXXXXX \=
XXXXXX \kill
11 \> 21 \> 31 \> 41 \> 51 \> 61 \> 71 \\
\footnotesize ALPHAG \>\footnotesize ALPHAP \>\footnotesize NEQMAX \>\footnotesize
XPLAS \>\footnotesize ZPLAS \>\footnotesize GZERO \>\footnotesize QZERO
\end{tabbing}
\setw{ALPHAG X}
\begin{tabbing}
ALPHAG \= X\= \parbox[t]{\width}{THIS IS JUNK} \kill
ALPHAG \> \> \parbox[t]{\width}{The initial toroidal field is given by $g\nabla \phi$ where} \index{toroidal field|(}\\
 \\
\> \>For IFUNC =1: \\
 \\
 \> \> $gg^{'}$ = [GP1*FF1($\Psi$)+GP2*FF2($\Psi$)]\\
  \\
 \> \> where\\
  \\
 \> \> FF1($\hat{\Psi})=-\hat{\Psi}^{\rm ALPHAG}$\\
 \> \> FF2($\hat{\Psi})=-4.0\hat{\Psi}^{\rm ALPHAG} [1-\hat{\Psi}]$\\
 \> \> $\hat{\Psi}$=($\Psi_{lim}-\Psi)/(\Psi_{lim}-\Psi_{min})$\\
 \\
 \> \> \parbox[t]{\width}{And GP1 and GP2 are determined so that the central q value is QZERO
and the total plasma current is PCUR(ISTART).}\\
 \\
\> \> For IFUNC=2:
\end{tabbing}
\begin{displaymath}
\frac{1}{2} \frac{dg^2}{d\Psi} = ({\rm XPLAS^2*PO*(1/BETAJ-1)}*\left[
\frac{e^{-({\rm ALPHAG})\hat{\Psi}}-e^{-({\rm ALPHAG})}}{e^{-({\rm ALPHAG})}-1} \right
]
\end{displaymath}
\begin{tabbing}
ALPHAG \= X\= \parbox[t]{\width}{THIS IS JUNK} \kill
\> \> \parbox[t]{\width}{where $\hat{\Psi} = (\Psi-\Psi_{min})/(\Psi_{lim}-\Psi_{min})$,
BETAJ=ACOEF(27), and PO above and in the pressure equation are initialized by the type 17
card, but are iterated (renormalized) so the total plasma current is PCUR(ISTART) and $g$=GZERO at
$\Psi=\Psi_{lim}$.}\\
 \\
\> \>For IFUNC=3:\\
 \\
\> if (DELG $>$ 0) $g$=GZERO + (const) $\hat{\Psi}^{\rm ALPHAG}$ \\
\> \> \parbox[t]{\width}{where (const) is chosen to make plasma current equal to PCUR(ISTART)
         (as specified on the type 16 card)}\\
\>if (DELG $<$ 0)  $g$=GPRFP*(1+(DELG-1)$\hat{\Psi}^{\rm ALPHAG}$)\\
 \\
\> \> \parbox[t]{\width}{where DELG=ACOEF(25) and GRPFP=ACOEF(26) is iterated
(renormalized) so the total plasma current is PCUR(ISTART)}\\
 \\
\> \> For IFUNC=4:\\
 \\
\> \>$g^2 = \rm GZERO^2 +2*GP1*FF1(\Psi)$\\
 \\
\> \> \parbox[t]{\width}{and GP1 is determined so the total plasma current is GCUR(1)}\\
 \\
\> \>For IFUNC=5:
\end{tabbing}
\begin{displaymath}
-\frac{1}{2} \frac{dg^2}{d\Psi}=\frac{{\rm GP1}/2 \pi \eta +(p^{'}+ \langle J_{CD} \rangle )
/ \langle R^{-2} \rangle}{ \langle B^2 \rangle / \langle B^2_T \rangle}\\
\end{displaymath}
\begin{tabbing}
ALPHAG \= X\= \parbox[t]{\width}{THIS IS JUNK} \kill
ALPHAP \> \>\parbox[t]{\width}{Pressure exponent for equilibrium calculation \index{plasma!pressure}(see type 02
card)}\\
NEQMAX \> \> \parbox[t]{\width}{Maximum number of equilibrium iterations allows. Normal
value is 200.  If NEQMAX is negative, the absolute value is used and the error flag is skipped
if convergence is not obtained in ABS(NEQMAX) iterations.}\\
XPLAS \> \> \parbox[t]{\width}{Initial guess for the $x$ coordinate of the magnetic axis.
This value is used as the nominal major radius in several calculations.} \\
ZPLAS \> \>\parbox[t]{\width}{Initial guess for the $z$ coordinate of the magnetic axis.
This value is used as the nominal vertical position in several calculations.} \\
GZERO \> \>\parbox[t]{\width}{Vacuum toroidal field given by GZERO $\nabla \phi$.  This can be specified
 as a function of time on the type 27 card.} \index{toroidal field|)}\\
QZERO \> \>\parbox[t]{\width}{Initial value of the safety factor at the magnetic axis for
IFUNC=1}
\end{tabbing}
\newpage \subsubsection{Card 14 - Initial Conditions 2}
\begin{tabbing}
XXXXXX \= XXXXXX \= XXXXXX \= XXXXXX \= XXXXXX \= XXXXXX \=
XXXXXX \kill
11 \> 21 \> 31 \> 41 \> 51 \\
\footnotesize ISTART \>\footnotesize XZERIC \>\footnotesize AXIC \>\footnotesize ZZERIC
\>\footnotesize BZIC 
\end{tabbing}
\setw{ISTART XX}
\begin{tabbing}
ISTART XX\= \parbox[t]{\width}{used to set tabs} \kill
ISTART \> \parbox[t]{\width}{This indicates at which time point TPRO(I) as specified on the
type 18 card the calculation is to begin.  The normal value is 1.} \index{time!starting}\\
XZERIC \>\parbox[t]{\width}{If this is nonzero, the initial equilibrium iteration will be initialized
with the plasma current distributed over a rectangular region centered at XZERIC and ZZERIC
and with half width AXIC and half height BZIC.  If these variables are specified, then the initial
plasma position XPLAS and ZPLAS on the type 13 card are overwritten.}\\
AXIC \> See above\\
ZZERIC \> See above\\
BZIC \> See above
\end{tabbing}
\newpage \subsubsection{Card 15 - Coil Group Current}
\index{coils!current}
\begin{tabbing}
XXXXXX \= XXXXXX \= XXXXXX \= XXXXXX \= XXXXXX \= XXXXXX \=
XXXXXX \kill
11 \> 21 \> 31 \> 41 \> 51 \> 61 \> 71\\
\footnotesize IGROUP \>\footnotesize GCUR(1)  \>\footnotesize GCUR(2) \>\footnotesize
GCUR(3) \>\footnotesize GCUR(4) \>\footnotesize GCUR(5) \>\footnotesize GCUR(6)
\end{tabbing}
\setw{IGROUP X}
\begin{tabbing}
IGROUP  X\= \parbox[t]{\width}{this is junk to set tabs} \kill
IGROUP \> \parbox[t]{\width}{The group number used to identify the coil. It is specified on type
9 and 10 cards for the external and internal coils.  If IGROUP$<$0, then ABS(IGROUP) is used
and this coil group has zero net current constraint applied if ACOEF(41)$>$0.  If
RESGS(IGROUP)
is non-zero, then this resistance is used for the group resistance.}\\
GCUR(I) \> \parbox[t]{\width}{The programmed coil current (kA) for the coil group IGROUP
at
time TPRO(I).  When using the Hofmann control scheme only the initial coil currents are
needed.}
\end{tabbing}
\pagebreak
\newpage \subsubsection{Card 16 - Plasma Current}
\index{plasma!current}
\begin{tabbing}
XXXXXX \= XXXXXX \= XXXXXX \= XXXXXX \= XXXXXX \= XXXXXX \=
XXXXXX \kill
11 \> 21 \> 31 \> 41 \> 51 \> 61 \> 71\\
\footnotesize  -  \>\footnotesize PCUR(1)  \>\footnotesize PCUR(2) \>\footnotesize PCUR(3)
\>\footnotesize PCUR(4)  \>\footnotesize PCUR(5) \>\footnotesize PCUR(6)
\end{tabbing}
\setw{PCUR(I) X}
\begin{tabbing}
PCUR(I) X\= \parbox[t]{\width}{this is junk to set tabs} \kill
PCUR(I) \> \parbox[t]{\width}{The programmed plasma current (kA) at the time TPRO(I)}
\end{tabbing}
\newpage \subsubsection{Card 17 - Plasma Pressure}
\index{plasma!pressure}
\begin{tabbing}
XXXXXX \= XXXXXX \= XXXXXX \= XXXXXX \= XXXXXX \= XXXXXX \=
XXXXXX \kill
11 \> 21 \> 31 \> 41 \> 51 \> 61 \> 71\\
\footnotesize  -  \>\footnotesize PPRES(1)  \>\footnotesize PPRES(2) \>\footnotesize PPRES(3)
\>\footnotesize PPRES(4)  \>\footnotesize PPRES(5) \>\footnotesize PPRES(6)
\end{tabbing}
\setw{PPRES(I) X}
\begin{tabbing}
PPRES(I) X\= \parbox[t]{\width}{this is junk to set tabs} \kill
PPRES(I) \> \parbox[t]{\width}{The programmed plasma pressure (mks) at the time TPRO(I).
For
IPRES=0, only the initial value is needed. For IPRES=1, all values are used.}
\end{tabbing}
\newpage \subsubsection{Card 18 - Time }
\index{time!specified points}
\begin{tabbing}
XXXXXX \= XXXXXX \= XXXXXX \= XXXXXX \= XXXXXX \= XXXXXX \=XXXXXX
\kill
11 \> 21 \> 31 \> 41 \> 51 \> 61 \> 71\\
\footnotesize  -  \>\footnotesize TPRO(1)  \>\footnotesize TPRO(2) \>\footnotesize TPRO(3)
\>\footnotesize TPRO(4)  \>\footnotesize TPRO(5) \>\footnotesize TPRO(6)
\end{tabbing}
\setw{TPRO(I) X}
\begin{tabbing}
TPRO(I) X\= \parbox[t]{\width}{this is junk to set tabs} \kill
TPRO(I) \> \parbox[t]{\width}{Time (in seconds) corresponding to GCUR(I),PCUR(I), etc.  The
intermediate values are linearly interpolated for ICUBE=0, cubic interpolation is used for
ICUBE=1 (set by ACOEF(17)).  Note that while most time dependent quantities are interpolated
between time points, auxiliary heating system powers specified on type 23 (Neutral Beam) and 46
 (Lower Hybrid) cards come on abruptly at these times and stay at the fixed level during each time
interval.}
\end{tabbing}
\newpage \subsubsection{Card 19 - Feedback 1}
\index{feedback!systems|(}
\begin{tabbing}
XXXXXX \= XXXXXX \= XXXXXXX \= XXXXXX \= XXXXXX \= XXXXXX \=XXXXXX
\kill
11 \> 21 \> 31 \> 41 \> 51 \> 61 \> 71\\
\footnotesize  L  \>\footnotesize NRFB(L)  \>\footnotesize NFEEDO(L) \>\footnotesize
FBFAC(L) \>\footnotesize FBCON(L)  \>\footnotesize IDELAY(L) \>\footnotesize FBFACI(L)
\end{tabbing}
\setw{NFEEDO(L) X}
\begin{tabbing}
NFEEDO(L) X\= SET TABS \kill
L \> Number of feedback system\\
NRFB(L) \> If NRFB(L)$>$0, indicates coil group number for feedback\\
        \> If NRFB(L)=0, indicates feedback on plasma current\\
NFEEDO(L) \> \parbox[t]{\width}{Observation pair number (type 8) used in feedback system.  
This is ignored if IPEXT(L) $>$ 3 (type 20).}\\
FBFAC(L) \> \parbox[t]{\width}{This is a proportionality factor between the coil group current
and the desired flux difference.  Units are (amps/weber/radian)  For external coils, this current
is changed instantaneously for ICIRC=0, or a voltage is applied through
the circuit equations for ICIRC=1.
For internal coils, a voltage is applied (proportional to the wire
resistivity) so that the desired current will be obtained after the coil L/R time. If IPEXT(L)=4 on
corresponding type 20 card, FBFAC(L) is the proportionality factor between coil group current
desired and difference between plasma current and plasma current desired.}\\
FBCON(L) \> \parbox[t]{\width}{Flux offset, FBFAC(L) multiplies :\\
(PSI1-PSI2-FBCON(L)*FAC)}\\
IDELAY(L) \> \parbox[t]{\width}{If this is greater than zero, a time delay of IDELAY(L) time
steps is introduced into the calculations.  Note that the parameter PDELAY must be greater than
the maximum IDELAY(L).}\\
FBFACI(L) \> \parbox[t]{\width}{This is the time integral feedback proportionality term.  It is
the same as FBFAC(L) except it multiplies the time integral of the flux or current difference. 
May be superimposed with FBFAC.}\\
 \\
{\bf Note} :\> \parbox[t]{\width}{If the first field on the type 19 card is equal to 1000.0, this
card defines time varying observation points for the feedback system defined by the preceding
type 19 card. The format is similar to that of the type [15,16,17,18,23,24] cards:}
\end{tabbing}
\begin{tabbing}
XXXXXX \= XXXXXXXXX \= XXXXXX \= XXXXXX \= XXXXXX \= XXXXXX
\=XXXXXX \kill
11 \> 21 \> 31 \> 41 \> 51 \> 61 \> 71\\
\footnotesize  1000.0  \>\footnotesize NFEEDV(1,L)  \>\footnotesize (2,L) \>\footnotesize (3,L)
\>\footnotesize (4,L)  \>\footnotesize (5,L) \>\footnotesize(6,L)
\end{tabbing}
\setw{NFEEDV(I,L) }
NFEEDV(I,L) \parbox[t]{\width}{Observation pair number (type 08) used in feedback system
L at time point I (type 18).  Multiple cards can be included to define more than 6 points}
\newpage \subsubsection{Card 20 - Feedback 2 }
\begin{tabbing}
XXXXXX \= XXXXXXX \= XXXXXXX \= XXXXXXX \= XXXXXXX \= XXXXXXX
\=XXXXXXX \kill
11 \> 21 \> 31 \> 41 \> 51 \> 61 \> 71\\
\footnotesize  L  \>\footnotesize TFBONS(L)  \>\footnotesize TFBOFS(L) \>\footnotesize
FBFAC1(L) \>\footnotesize FBFACD(L) \>\footnotesize IPEXT(L) 
\end{tabbing}
\setw{TFBONS(L) = 1.0 }
\begin{tabbing}
TFBONS(L) \= = 1.0XX \=  SET TABS \kill
L \> \> \parbox[t]{\width}{Number of feedback system (same as that on corresponding type 19
card)}\\
TFBONS(L) \> \> \parbox[t]{\width}{Time when feedback system L is turned on (sec)}\\
TFBOFS(L) \> \> \parbox[t]{\width}{Time when feedback system L is turned off (sec)}\\
FBFAC1(L) \> \> \parbox[t]{\width}{If $>$0, factor multiplying FBCON is proportional to the
(plasma current)/(final current)}\\
FBFACD(L) \> \> \parbox[t]{\width}{This is the time derivative feedback proportionality term. 
It is the same as FBFAC(L) and FBFACI(L) except it multiplies the time derivative of the flux
or current difference.  It may be superimposed with FBFAC and FBFACI}\\
IPEXT(L) \> \>\parbox[t]{\width}{Signifies which flux to used from the observation coils}\\
\> = 1.0 \> Total flux per radian\\
\> = 2.0 \> Flux from coils only (not presently available)\\
\> = 3.0 \> Flux from plasma only (not presently available)\\
\> = 4.0 \> \parbox[t]{\width}{Feedback signal is proportional to plasma current minus preprogrammed
value.  For this option , FBFAC is dimensionless}\\
\> = 5.0 \> \parbox[t]{\width}{Feedback signal is proportional to (XMAG-XMAGO(t)). where
XMAGO(t) is defined on type 30 card$\ldots$see note 4}\\
\> = 6.0 \> \parbox[t]{\width}{Feedback signal is proportional to (ZMAG-ZMAGO(t)). where
ZMAGO(t) is defined on type 31 card$\ldots$see note 4}\\
\> = 7-10 \> \parbox[t]{\width}{Feedback is proportional to EPS1C-EPS4C}\\
\> = 10NN \> \parbox[t]{\width}{Feedback is proportional to current in wire NN}\\
\> = 21-24 \> \parbox[t]{\width}{Special option for RFP, proportional to $\cos(\theta) -
\cos(4\theta)$}\\
\> = 25 \> \parbox[t]{\width}{Special shape control using acoef(510) and cards 62-63
}\\
\end{tabbing}
{\bf Note 1} : If TFBONS or TFBOFS are negative, then their absolute value refers to the cycle
number for which the feedback is turned on or off.\\
 \\
{\bf Note 2} : If controlling plasma current by using IPEXT(L)=4, the automatic plasma current
control should be turned off by setting ACOEF(11)=0.\\
 \\
{\bf Note 3} : If IPEXT(L) = 7,8,9,10, the switch ISVD must be set to 1.0 on type 03 card\\
 \\
{\bf Note 4} : Feedback signal multiplied by ($I_p$/1 MA) for IPEXT(5) or IPEXT(6) \index{feedback!systems|)}
\pagebreak
\subsubsection{Card 21 - Contour Plots}
\index{output!contour plots}
\begin{tabbing}
XXXXXX \= XXXXXX \= XXXXXX \= XXXXXX \= XXXXXX \= XXXXXX \=
XXXXXX \kill
11 \> 21 \> 31 \> 41 \> 51 \> 61 \> 71\\
\footnotesize  ICPLET  \>\footnotesize ICPLGF  \>\footnotesize ICPLWF \>\footnotesize ICPLPR
\>\footnotesize ICPLBV 
\>\footnotesize ICPLUV \>\footnotesize ICPLXP
\end{tabbing}
If any of these switches are set to 1.0, the following contour plots are produced every NSKIPL
cycles.
\begin{tabbing}
ICPLWF xx \= This is junk to set tabs \kill
ICPLET    \> Resistivity array ETAY\\
        \> If (IRFP=1) ETA*J\\
ICPLGF \> Toroidal field function g\\
ICPLWF \> Toroidal velocity W \\
       \> If (IRFP=1) ( $\vec{J} \cdot \vec{B} /B^2$ ) \\
ICPLPR \> Pressure p\\
ICPLBV \> Curl of the velocity field B $\equiv {\it \Delta}^* A$\\
       \> If (IRFP=1) HYPER/J \\
ICPLUV \> Divergence of velocity field U$\equiv \nabla^2 \Omega$\\
       \> If (IRFP=1) (ETA*J+HYPER)/(ETA*J)\\
ICPLXP \> Close-up of poloidal flux near x-point region
\end{tabbing}
\newpage \subsubsection{Card 22 - Vector Plots}
\index{output!vector plots}
\begin{tabbing}
XXXXXX \= XXXXXX \= XXXXXX \= XXXXXX \= XXXXXX \= XXXXXX \=
XXXXXX \kill
11 \> 21 \> 31 \> 41 \> 51 \> 61 \> 71\\
\footnotesize  IVPLBP  \>\footnotesize IVPLVI  \>\footnotesize IVPLFR \>\footnotesize IVPLJP
\>\footnotesize IVPLVC 
\>\footnotesize IVPLVT \>\footnotesize -
\end{tabbing}
If any of these switches are set to 1.0, the following vector plots are produced every NSKIPL
cycles.
\begin{tabbing}
IVPLFR xx \= Set the tab stops \kill
IVPLBP \> Poloidal magnetic field\\
IVPLVI \> Incompressible velocity field\\
IVPLFR \> Forces\\
IVPLJP \> Poloidal current\\
IVPLVC \> Compressible velocity field\\
IVPLVT \> Total velocity field\\
\end{tabbing}
\newpage \subsubsection{Card 23 - Neutral Beam}
\index{neutral beam!amplitude}
\begin{tabbing}
XXXXXX \= XXXXXX \= XXXXXX \= XXXXXX \= XXXXXX \= XXXXXX \=
XXXXXX \kill
11 \> 21 \> 31 \> 41 \> 51 \> 61 \> 71\\
\footnotesize  -  \>\footnotesize BEAMP(1)  \>\footnotesize BEAMP(2) \>\footnotesize
BEAMP(3) \>\footnotesize BEAMP(4) 
\>\footnotesize BEAMP(5) \>\footnotesize BEAMP(6)
\end{tabbing}
\setw{BEAMP(I) X}
\begin{tabbing}
BEAMP(I) X\= \parbox[t]{\width}{this is junk to set tabs} \kill
BEAMP(I) \> \parbox[t]{\width}{The amplitude of the neutral beam source (MW) at time
TPRO(I).  The deposition profile is given on the type 25 card.}
\end{tabbing}
\newpage \subsubsection{Card 24 - Plasma Density}
\index{plasma!density}
\begin{tabbing}
XXXXXX \= XXXXXX \= XXXXXX \= XXXXXX \= XXXXXX \= XXXXXX \=
XXXXXX \kill
11 \> 21 \> 31 \> 41 \> 51 \> 61 \> 71\\
\footnotesize  -  \>\footnotesize RNORM(1)  \>\footnotesize RNORM(2) \>\footnotesize
RNORM(3) \>\footnotesize RNORM(4) \>\footnotesize RNORM(5) \>\footnotesize RNORM(6)
\end{tabbing}
\setw{RNORM(I) X}
\begin{tabbing}
RNORM(I) X\= \parbox[t]{\width}{this is junk to set tabs} \kill
RNORM(I) \> \parbox[t]{\width}{The normalized central density for IDENS=1 at time TPRO(I).
The actual density is RNORM(I)*DCGS.}
\end{tabbing}
\newpage \subsubsection{Card 25 - Neutral Beam Deposition Profile}
\index{neutral beam!deposition profile}
\begin{tabbing}
XXXXXX \= XXXXXX \= XXXXXX \= XXXXXXX \= XXXXXX \= XXXXXX \=
XXXXXX \kill
11 \> 21 \> 31 \> 41 \> 51 \> 61 \> 71\\
\footnotesize  ABEAM  \>\footnotesize DBEAM  \>\footnotesize NEBEAM \>\footnotesize
EBEAMKEV \>\footnotesize AMBEAM  \>\footnotesize FRACPAR \>\footnotesize IBOOTST
\end{tabbing}
\setw{EBEAMKEV X}
\begin{tabbing}
EBEAMKEV X\= \parbox[t]{\width}{This is junk to set tabs} \kill
ABEAM \>\parbox[t]{\width}{This variable along with DBEAM and NEBEAM specify the
spatial
external heat source deposition profile which is multiplied by the beam amplitude parameter on
the type 23 card. (0.25)  The spatial form factor is}\\
 \\
\> FF=F1*F2/SUM\\
 \\
\> F1=$\rm DBEAM ^2/[(\hat{\Psi}-ABEAM)^2+DBEAM^2]$\\
 \\
\> F2=(1-$\rm \hat{\Psi}^2)^{NEBEAM}$\\
 \\
\>\parbox[t]{\width}{with $\hat{\Psi}=(\Psi-\Psi_{min})/(\Psi_{lim}-\Psi_{min})$ and SUM is
the normalization factor.}\\
\\
DBEAM \> See ABEAM above. (0.1)\\
NEBEAM \> See ABEAM above. (1.0)\\
EBEAMKEV \> Energy of the neutral beam ions in keV. (80.)\\
AMBEAM \> Mass of the neutral beam particles in amu. (1.0)\\
FRACPAR \> \parbox[t]{\width}{Fraction of beam particles which are oriented parallel to the
plasma current (0.0). -1$<$ FRACPAR $<$ 1. This can be input as a
function of time, TPRO(I), on the type 50 card.}\\
IBOOTST \>\parbox[t]{\width}{If IBOOTST$\ne$0, the bootstrap current\index{bootstrap current} is included in the
calculation.  If IBOOTST=1, the collisionless Hirshman model is used and if IBOOTST=2 the
collisional Harris model is used}       
\end{tabbing}
\pagebreak
\subsubsection{Card 26 - Anomalous Transport}
\index{transport!anomalous}
\begin{tabbing}
XXXXX \= XXXXXXX \= XXXXXXX \= XXXXXXX \= XXXXXXX \= XXXXXXX \=
XXXXXX \kill
11 \> 21 \> 31 \> 41 \> 51 \> 61 \> 71\\
\footnotesize  - \>\footnotesize FBCHIA(1)  \>\footnotesize FBCHIA(2) \>\footnotesize
FBCHIA(3) \>\footnotesize FBCHIA(4) \>\footnotesize FBCHIA(5) \>\footnotesize FBCHIA(6)
\end{tabbing}
\setw{FBCHIA(I) X}
\begin{tabbing}
FBCHIA(I) X\= \parbox[t]{\width}{This is junk to set tabs} \kill
FBCHIA(I) \>Factor by which thermal conductivity is enhanced at time TPRO(I) \\
\end{tabbing}
\newpage \subsubsection{Card 27 - Toroidal Field}
\index{toroidal field}
\begin{tabbing}
XXXXX \= XXXXXX \= XXXXXX \= XXXXXX \= XXXXXX \= XXXXXX \= XXXXXX \kill
11 \> 21 \> 31 \> 41 \> 51 \> 61 \> 71\\
\footnotesize  - \>\footnotesize GZEROV(1)  \>\footnotesize $\ldots$(2) \>\footnotesize  $\ldots$(3) \>\footnotesize $\ldots$(4) \>\footnotesize $\ldots$(5) \>\footnotesize GZEROV(6)
\end{tabbing}
\setw{GZEROV(I) X}
\begin{tabbing}
GZEROV(I) X\= \parbox[t]{\width}{This is junk to set tabs} \kill
GZEROV(I) \> Vacuum toroidal field function GZERO at time TPRO(I) \\
\end{tabbing}

\newpage \subsubsection{Card 28 - Loop Voltage}
\index{loop voltage}
\begin{tabbing}
XXXXX \= XXXXXXX \= XXXXXX \= XXXXXX \= XXXXXX \= XXXXXX \= XXXXXX \kill
11 \> 21 \> 31 \> 41 \> 51 \> 61 \> 71\\
\footnotesize  - \>\footnotesize VLOOPV(1)  \>\footnotesize $\ldots$(2) \>\footnotesize  $\ldots$(3) \>\footnotesize $\ldots$(4) \>\footnotesize $\ldots$(5) \>\footnotesize
VLOOPV(6)
\end{tabbing}
\setw{VLOOPV(I) X}
\begin{tabbing}
VLOOPV(I) X\= \parbox[t]{\width}{This is junk to set tabs} \kill
VLOOPV(I) \> \parbox[t]{\width}{Programmed loop voltage for OH system at time TPRO(I). 
In
general, a loop voltage determined by feedback will be superimposed on VLOOPV(I).  The
``automatic'' plasma current control feedback is proportional to ACOEF(11). The maximum and
minimum loop voltages (sum of preprogrammed and feedback) are limited by ACOEF(15)(min)
and ACOEF(16)(max).  Currents in passive conductors are initialized when
VLOOPV(ISTART)$>$0 and ACOEF(41)=0.}
\end{tabbing}
\newpage \subsubsection{Card 29 - PEST Output}
\index{PEST file}
\begin{tabbing}
XXXXXX \= XXXXXX \= XXXXXX \= XXXXXX \= XXXXXX \= XXXXXX \=
XXXXXX \kill
11 \> 21 \> 31 \> 41 \> 51 \> 61 \> 71\\
\footnotesize  - \>\footnotesize TPEST(1)  \>\footnotesize TPEST(2) \>\footnotesize TPEST(3)
\>\footnotesize TPEST(4) \>\footnotesize TPEST(5) \>\footnotesize TPEST(6)
\end{tabbing}
\setw{TPEST(I) X}
\begin{tabbing}
TPEST(I) X\= This is junk to set tabs \kill
TPEST(I) \> \parbox[t]{\width}{The specified times at which PEST output is to be written onto
file EQDSKA for IPEST=1. Plots and a restart file are also written at these times.}
\end{tabbing}

\newpage \subsubsection{Cards 30 and 31 - Magnetic Axis}
\index{feedback!systems}
\begin{tabbing}
XXXXXX \= XXXXXX \= XXXXXX \= XXXXXX \= XXXXXX \= XXXXXX \=
XXXXXX \kill
11 \> 21 \> 31 \> 41 \> 51 \> 61 \> 71\\
\footnotesize  - \>\footnotesize XMAGO(1)  \>\footnotesize XMAGO(2) \>\footnotesize
XMAGO(3) \>\footnotesize XMAGO(4) \>\footnotesize XMAGO(5) \>\footnotesize XMAGO(6)\\
\footnotesize  - \>\footnotesize ZMAGO(1)  \>\footnotesize ZMAGO(2) \>\footnotesize
ZMAGO(3) \>\footnotesize ZMAGO(4) \>\footnotesize ZMAGO(5) \>\footnotesize ZMAGO(6)
\end{tabbing}
\setw{XMAGO(I) X}
\begin{tabbing}
XMAGO(I) X\=  set tabs \kill
XMAGO(I)  \=  \parbox[t]{\width}{The $x$-magnetic axis position corresponding to time
TPRO(I) for use in feedback system (type 19,20) with IPEXT=5}\\
ZMAGO(I)  \=  \parbox[t]{\width}{The $z$-magnetic axis position corresponding to time
TPRO(I) for use in feedback system (type 19,20) with IPEXT=6}
\end{tabbing}

\newpage \subsubsection{Card 32 - Divertor Plate}
\index{divertor}
\begin{tabbing}
XXXXX \= XXXXXXX \= XXXXX \= XXXXX \= XXXXX \= XXXXXXX \=  XXXXXX \kill
11 \> 21 \> 31 \> 41 \> 51 \> 61 \> 71\\
\footnotesize N \>\footnotesize XLPLATE(N) \>\footnotesize ZL$\ldots$(N) \>\footnotesize
XR$\ldots$(N) \>\footnotesize ZR$\ldots$(N) \>\footnotesize FPLATE(N,1) \>\footnotesize FPLATE(N,2)
\end{tabbing}
\setw{XLPLATE(N) X}
\begin{tabbing}
XLPLATE(N) X\= this is to set tabs \kill
N \> Number of divertor plate\\
XLPLATE(N) \> \parbox[t]{\width}{The $x$-coordinate of leftmost side of divertor plate N.}\\
ZLPLATE(N) \> \parbox[t]{\width}{The $z$-coordinate of leftmost side of divertor plate N.}\\
XRPLATE(N) \> \parbox[t]{\width}{The $x$-coordinate of rightmost side of divertor plate N.}\\
ZRPLATE(N) \> \parbox[t]{\width}{The $z$-coordinate of rightmost side of divertor plate N.}\\
FPLATE(N,1) \> \parbox[t]{\width}{Fraction of charged particle heat flux deposited on divertor
plate N. Outside strike point is 1, inside is 2}\\
FPLATE(N,2) \> \parbox[t]{\width}{See FPLATE(N,1)}\\
 \\
\parbox[t]{\textwidth}{The plate will be divided into PNSEG bins, and the heat flux in each bin
will be calculated and plotted. One sided exponential distributions are used, based on midplane
scrapeoff distance of 0.6 cm.}\\
\\
{\bf Note} : \> \parbox[t]{\width}{The default divertor shape follows a straight line between the coordinates 
specified here.  If this card is followed by additional type 32 cards with option 1000 in the second field,
additional defining points are added.  The individual PNSEG+1 x-z coordinates are input 3 per
card as follows}\\
\\
{\bf Note} : \> \parbox[t]{\width}{Acoef(700)-(704) must be defined for divertor plots}\\
\\
XXXXX \= XXXXX \= XXXXX \= XXXXX \= XXXXX \= XXXXX \= XXXXX \= XXXXX
\kill
32 \> 10000. \> X(I) \> Z(I) \> X(I+1) \> Z(I+1) \> X(I+2) \> Z(I+2)
\end{tabbing}
\pagebreak
\subsubsection{Card 33 - Gap Resistance}
\index{gap resistance}
\begin{tabbing}
XXXXXX \= XXXXXX \kill
11 \> 21 \\
\footnotesize IGROUP \>\footnotesize RESGS(IGROUP)
\end{tabbing}
\setw{IGROUP X}
\begin{tabbing}
IGROUP X\= this is to set tabs \kill
IGROUP \> \parbox[t]{\width}{Group number of coil (same as type 15 card).}\\
RESGS \> \parbox[t]{\width}{The resistance of gap in coil.  This will override the gap resistance
computed from ACOEF(41) when IGROUP is negative.}
\end{tabbing}

\newpage \subsubsection{Card 34 - Vacuum Temperature}
\index{vacuum temperature} \index{temperature!vacuum}
\begin{tabbing}
XXXXXX \= XXXXXX \= XXXXXX \= XXXXXX \= XXXXXX \= XXXXXX \=
XXXXXX \kill
11 \> 21 \> 31 \> 41 \> 51 \> 61 \> 71\\
\footnotesize  -  \>\footnotesize TEVVO(1)  \>\footnotesize TEVVO(2) \>\footnotesize
TEVVO(3) \>\footnotesize TEVVO(4) \>\footnotesize TEVVO(5) \>\footnotesize TEVVO(6)
\end{tabbing}
\setw{TEVVO(I) X}
\begin{tabbing}
TEVVO(I) X\= set the tab \kill
TEVVO(I) \> \parbox[t]{\width}{Vacuum temperature TEVV at time point I. This
card overrides the value specified on type 12 card.  This option is NOT
recommended, rather set TEVV negative on the type 12 card and allow TSC
to determine the value}
\end{tabbing}
\newpage \subsubsection{Card 35 - Mass Enhancement}
\index{mass enhancement}
\begin{tabbing}
XXXXXX \= XXXXXX \= XXXXXX \= XXXXXX \= XXXXXX \= XXXXXX \=
XXXXXX \kill
11 \> 21 \> 31 \> 41 \> 51 \> 61 \> 71\\
\footnotesize  -  \>\footnotesize FFACO(1)  \>\footnotesize FFACO(2) \>\footnotesize FFACO(3)
\>\footnotesize TEVVO(4) \>\footnotesize FFACO(5) \>\footnotesize FFACO(6)
\end{tabbing}
\setw{FFACO(I) X}
\begin{tabbing}
FFACO(I) X\= set the tab \kill
FFACO(I) \>\parbox[t]{\width}{Mass enhancement FFAC at time point I.  Inclusion of of this card
will override the value specified on type 03 card.}
\end{tabbing}
\newpage \subsubsection{Card 36 - Resistivity Enhancement}
\index{resistivity enhancement}
\begin{tabbing}
XXXXXX \= XXXXXX \= XXXXXX \= XXXXXX \= XXXXXX \= XXXXXX \=
XXXXXX \kill
11 \> 21 \> 31 \> 41 \> 51 \> 61 \> 71\\
\footnotesize  -  \>\footnotesize ZEFFV(1)  \>\footnotesize ZEFFV(2) \>\footnotesize ZEFFV(3)
\>\footnotesize ZEFFV(4) \>\footnotesize ZEFFV(5) \>\footnotesize ZEFFV(6)
\end{tabbing}
\setw{ZEFFV(I) X}
\begin{tabbing}
ZEFFV(I) X\= set the tab \kill
ZEFFV(I) \> \parbox[t]{\width}{Resistivity enhancement ZEFF at time point I.  Inclusion of this
card will override the value specified on type 12 card.  Only used for IIMP=0 on type 07.}
\end{tabbing}

\newpage \subsubsection{Card 37 - Voltage Group}
\begin{tabbing}
XXXXXX \= XXXXXX \= XXXXXX \= XXXXXX \= XXXXXX \= XXXXXX \=
XXXXXX \kill
11 \> 21 \> 31 \> 41 \> 51 \> 61 \> 71\\
\footnotesize  IGROUP \>\footnotesize GVOLT(1)  \>\footnotesize GVOLT(2) \>\footnotesize
GVOLT(3) \>\footnotesize GVOLT(4) \>\footnotesize GVOLT(5) \>\footnotesize GVOLT(6)
\end{tabbing}
\setw{GVOLT(I) X}
\begin{tabbing}
GVOLT(I) X\= set tabs \kill
GVOLT(I) \> \parbox[t]{\width}{The preprogrammed voltage (kV) for coil group IGROUP at
time TPRO.  This is the equivalent one turn voltage.}
\end{tabbing}

\newpage \subsubsection{Card 38 - ILHCD}
\index{heating!RF}
\begin{tabbing}
XXXXXX \= XXXXXX \= XXXXXX \= XXXXXX \= XXXXXX \= XXXXXX \=
XXXXXX \kill
11 \> 21 \> 31 \> 41 \> 51 \> 61 \> 71\\
\footnotesize  ILHCD \>\footnotesize VILIM  \>\footnotesize FREQLH \>\footnotesize AION
\>\footnotesize ZION \>\footnotesize CPROF \>\footnotesize IFK\\
\end{tabbing}
\setw{ILHCD = 0.0 XX}
\begin{tabbing}
ILHCD \= 0.0 XX\= set tabs \kill
ILHCD \>= 0.0 \> \parbox[t]{\width}{No LHCD calculation and no hot plasma conductivity correction}\\

      \> = 1.0 \> \parbox[t]{\width}{LHCD calculation and hot plasma conductivity contribution are
included}\\
VILIM \> \>  \parbox[t]{\width}{Lower velocity limit for the LHCD spectrum normalized to
local thermal velocity. (typical value 2 to 3)}\\
FREQLH \> \> \parbox[t]{\width}{Frequency in GHz of the LH wave(3.7 for instance)}\\
AION \> \>\parbox[t]{\width}{Ratio of masses $\rm m_i/M_p$ for the dominant ion species (1
for hydrogen)}\\
ZION \> \>\parbox[t]{\width}{Atomic number of the dominant ion species}\\
CPROF \> \>\parbox[t]{\width}{Option to calculate the RF current profile}\\
      \> = 0.0 \>\parbox[t]{\width}{RF current profile is calculated from the Fisch formula
(depends on power)}\\
      \> = 1.0 \>\parbox[t]{\width}{RF current profile is calculated independently of power, from
cards 55-58 according to}
\end{tabbing}
\begin{displaymath}
\> \> \frac{d_c^2 r^{a_{c_1}}(1-r)^{a_{c_2}}}{(r-a_c)^2+d_c^2}\\
\end{displaymath}
\begin{tabbing}
ILHCD \= 0.0 XX\= set tabs \kill
IFK \>= 1.0 \> Read data file TSCOUTA\\
    \>= 2.0 \> Call LSC(see acoef(700),(701),(106))
\end{tabbing}

\newpage \subsubsection{Card 39 - External Coils 2}
\index{coils!external}
\begin{tabbing}
XXXXXX \= XXXXXX \= XXXXXX \= XXXXXX \= XXXXXX \= XXXXXX \=
XXXXXX \kill
11 \> 21 \> 31 \> 41 \> 51 \> 61 \> 71\\
\footnotesize  ICO \>\footnotesize DXCOIL  \>\footnotesize DZCOIL \>\footnotesize FCU
\>\footnotesize FSS \>\footnotesize TEMPC \>\footnotesize CCICS \\
\end{tabbing}
\setw{DXCOIL X} 
\begin{tabbing}
DXCOIL X\= set tabs \kill
ICO \> External coil number (same as on type 09 card)\\
DXCOIL \> Radial thickness of coil in meters\\
DZCOIL \> Vertical thickness of coil in meters\\
FCU \> \parbox[t]{\width}{Fraction of coil volume which is copper (see note below)}\\
FSS \> Fraction of coil volume which is stainless steel\\
TEMPC \> Initial temperature of coil in ${ }^{\footnotesize \circ}$K\\
CCICS \> Initial induced current in coil (kA)\\
 \\
Note: \> \parbox[t]{\width}{If FCU(N)$>$1, the truncated integer refers to the alloy type, while
the decimal fraction refers to the fraction}\\
Note:  \> \parbox[t]{\width}{For superconducting coils,
set ITEMP=-1 (type 12), FSS=0, TEMPC=2., FCU=0, RSCOILS=1.e-12 }\\
 \\
 \> 0.000$<$FCU$<$0.999 \= OFHC Copper\\
\>  1.000$<$FCU$<$1.999 \> AL25 (Glidcop)\\
\>  2.000$<$FCU$<$2.999 \> Berylium Copper
\end{tabbing}

\newpage \subsubsection{Card 40 - Output Reduction}
\index{output!reduction}
\begin{tabbing}
XXXXXXX \= XXXXXX \= XXXXXX \= XXXXXX \= XXXXXX \= XXXXXX \=
XXXXXX \kill
11 \> 21 \> 31 \> 41 \> 51 \> 61 \> 71\\
\footnotesize NOPLOT(1) \>\footnotesize $\ldots$(2) \>\footnotesize $\ldots$(3) \>\footnotesize
$\ldots$(4) \>\footnotesize $\ldots$(5) \>\footnotesize $\ldots$(6) \>\footnotesize NOPLOT(7)\\
\end{tabbing}
Plots are suppressed if the following numbers are assigned to the NOPLOT array on type(40) cards.
\begin{tabbing}
XX\= XXXXX \= set tabs \kill
\underline{NOPLOT} \> \>  \underline{Description} \\
\>1 \> Grid, coils and limiters \\
\>2 \> Switch and time step information\\
\>3 \> Filament growth rate model\\
\>4 \> Initial coil and wire information\\
\>5 \> Coil currents,cycle=\#\\
\>6 \> Current and flux\\
\>7  \> Special for spheromak formation \\
\>8 \> Special x-point plot\\
\>9 \> Heat flux, plate \# cycle \#\\
\>10 \> Profile plots(eg: q-prof vs poloidal flux, etc.)\\
\>11 \> Surface profiles, cycle=\#\\
\>12 \> Summary plot\\
\>13 \> Flux measurements of observation pairs\\
\>14 \> Special divertor plots\\
\>15 \> Group number current and voltage\\
\>16 \> Current groups\\
\>17 \> Group voltage\\
\>18 \> Group power\\
\>19 \> Group energy\\
\>20 \> Total power and energy\\
\>21 \> Coil temperature\\
\>22 \> Currents(kA)\\
\>23 \> Timing information\\
\>51 \> AMACH and EKIN vs time\\
\>52 \> IPLIM and ZMAG vs time, XMAG vs TIME and ZMAG\\
\>53 \> XMAG and CUR vs time\\
\>54 \> DELP.TPI and PMIN.TPI vs time\\
\>55 \> DIAMAG and SURFVOLT vs time\\
\>56 \> QZERO and QEDGE vs time\\
\>57 \> DT and BETA vs TIME\\
\>58 \> $\langle \rm N \rangle$/NMUR vs time and 1/q nr/B\\
\>   \> Density and INT ENER vs time\\
\>59 \> LI/2 vs time and LI vs q\\
\>60 \> TAUE-KG and TAU(MS) vs time\\
\>61 \> TI(0) and TE/TE-AV vs time\\
\>62 \> CHIOHMS and HFLUX-MW vs time\\
\>63 \> RO and MINORRAD vs time\\
\>64 \> DELT-TRI and ELLIP vs time\\
\>65 \> XSEP and ZSEP vs time\\
\>66 \> RESV-SEC and VSEC-TOT vs time\\
\>67 \> LOOPV-OH and VSEC-OH vs time\\
\>68 \> PTOT(MW) and PSEPCAL vs time\\
\>   \> Power flow in system\\
\>   \> FFAC and TEVV vs time\\
\>   \> NPSIT and RESID vs time\\
\>69 \> Nullapole and dipole vs time\\
\>70 \> Quadrupole and hexapole vs time\\
\>71 \> Octapole and decapole vs time
\end{tabbing}
{\bf Note} : To cancel the suppression of a certain plot, restart the job and input a negative number (eg: -9 ) \\
 \\
\underline{Example:}
\begin{tabbing}
XXXXXX \= XXXXXX \= XXXXXX \= XXXXXX \= XXXXXX \= XXXXXX \= XXXXXX
\kill
10. \> 2. \> +23. \> -9. \> 15. \> 12. \> 7.
\end{tabbing}
\pagebreak
\subsubsection{Card 41 - TF Ripple}
\index{toroidal field!ripple calculation}
\begin{tabbing}
XXXXXX \= XXXXXX \= XXXXXX \= XXXXXX \= XXXXXX \= XXXXXX \= XXXXXX
\kill
11 \> 21 \> 31 \> 41 \> 51 \> 61 \> 71\\
\footnotesize  IRIPPL \>\footnotesize NTFCOIL \>\footnotesize RIPMAX \>\footnotesize
RTFCOIL \>\footnotesize NPITCH \>\footnotesize RIPMULT \>\footnotesize IRIPMOD \\
\end{tabbing}
\setw{IRIPPLE = 0.0XX}
\begin{tabbing}
IRIPPLE \= 0.0XX\= SET TABS \kill
IRIPPLE \> = 0.0 \>Does not calculate ripple losses\\
        \> = 1.0 \> \parbox[t]{\width}{Does calculate ripple losses. {\em WARNING} : may be
expensive for time dependent calculation}\\
NTFCOIL \> \> Number of TF coils\\
RIPMAX  \> \> Ripple magnitude at radius of TF coil\\
RTFCOIL \> \> Radius of TF coil\\
NPITCH \> \>Number of pitch angles for integration\\
RIPMULT \> \>Ripple multiplier\\ 
IRIPMOD \>= 1.0\>CIT 2.1 meter design (U. Christenson)\\
        \>= 2.0\> TFTR model\\
        \>= 3.0\> Model RIPMAX*(R/RTFCOIL$)^{\rm NTFCOIL}$
\end{tabbing}
\newpage \subsubsection{Card 42 - Major Radius}
\index{radius!major}
\begin{tabbing}
XXXXXX \= XXXXXX \= XXXXXX \= XXXXXX \= XXXXXX \= XXXXXX \=
XXXXXX \kill
11 \> 21 \> 31 \> 41 \> 51 \> 61 \> 71\\
\footnotesize - \>\footnotesize  RZERV(1) \>\footnotesize RZERV(2) \>\footnotesize RZERV(3)
\>\footnotesize RZERV(4) \>\footnotesize RZERV(5) \>\footnotesize RZERV(6) \\
\end{tabbing}
\setw{RZERV(I) X}
\begin{tabbing}
RZERV(I) X\= set tab \kill
RZERV(I) \> \parbox[t]{\width}{The preprogrammed major radius at time TPRO(I)
for use in the Hofmann plasma shape control algorithm.}\\
\end{tabbing}
\newpage \subsubsection{Card 43 - Minor Radius}
\index{radius!minor}
\begin{tabbing}
XXXXXX \= XXXXXX \= XXXXXX \= XXXXXX \= XXXXXX \= XXXXXX \=
XXXXXX \kill
11 \> 21 \> 31 \> 41 \> 51 \> 61 \> 71\\
\footnotesize - \>\footnotesize  AZERV(1) \>\footnotesize AZERV(2) \>\footnotesize AZERV(3)
\>\footnotesize AZERV(4) \>\footnotesize AZERV(5) \>\footnotesize AZERV(6) \\
\end{tabbing}
\setw{AZERV(I) X}
\begin{tabbing}
AZERV(I) X\= set tab \kill
AZERV(I) \> \parbox[t]{\width}{The preprogrammed minor radius at time TPRO(I) f
or use in the Hofmann plasma shape control algorithm.}\\
\end{tabbing}
\pagebreak
\subsubsection{Card 44 - Ellipticity}
\index{ellipticity}
\begin{tabbing}
XXXXXX \= XXXXXX \= XXXXXX \= XXXXXX \= XXXXXX \= XXXXXX \=
XXXXXX \kill
11 \> 21 \> 31 \> 41 \> 51 \> 61 \> 71\\
\footnotesize - \>\footnotesize  EZERV(1) \>\footnotesize EZERV(2) \>\footnotesize EZERV(3)
\>\footnotesize EZERV(4) \>\footnotesize EZERV(5) \>\footnotesize EZERV(6) \\
\end{tabbing}
\setw{EZERV(I) X}
\begin{tabbing}
EZERV(I) X\= set tab \kill
EZERV(I) \> \parbox[t]{\width}{The preprogrammed ellipticity at time TPRO(I)
for use in the Hofmann plasma shape control algorithm.}\\
\end{tabbing}
\newpage \subsubsection{Card 45 - Triangularity}
\index{triangularity}
\begin{tabbing}
XXXXXX \= XXXXXX \= XXXXXX \= XXXXXX \= XXXXXX \= XXXXXX \=
XXXXXX \kill
11 \> 21 \> 31 \> 41 \> 51 \> 61 \> 71\\
\footnotesize - \>\footnotesize  DZERV(1) \>\footnotesize DZERV(2) \>\footnotesize DZERV(3)
\>\footnotesize DZERV(4) \>\footnotesize DZERV(5) \>\footnotesize DZERV(6) \\
\end{tabbing}
\setw{DZERV(I) X}
\begin{tabbing}
DZERV(I) X\= set tab \kill
DZERV(I) \> \parbox[t]{\width}{The preprogrammed triangularity at time TPRO(I)
for use in the Hofmann plasma shape control algorithm.}
\end{tabbing}
\newpage \subsubsection{Card 46 - Lower Hybrid Heating}
\index{heating!lower hybrid}
\begin{tabbing}
XXXXXX \= XXXXXX \= XXXXXX \= XXXXXX \= XXXXXX \= XXXXXX \=
XXXXXX \kill
11 \> 21 \> 31 \> 41 \> 51 \> 61 \> 71\\
\footnotesize - \>\footnotesize  PLHAMP(1) \>\footnotesize $\ldots$(2) \>\footnotesize $\ldots$(3)
\>\footnotesize $\ldots$(4) \>\footnotesize $\ldots$(5) \>\footnotesize PLHAMP(6) \\
\end{tabbing}
\setw{PLHAMP(I) X}
\begin{tabbing}
PLHAMP(I) X\= set tab \kill
PLHAMP(I) \> \parbox[t]{\width}{The lower hybrid heating power (MW) at time TPRO(I)}
\end{tabbing}
\newpage \subsubsection{Card 47 - Density Exponent 1}
\index{density profile}
\begin{tabbing}
XXXXXX \= XXXXXXX \= XXXXXX \= XXXXXX \= XXXXXX \= XXXXXX \=
XXXXXX \kill
11 \> 21 \> 31 \> 41 \> 51 \> 61 \> 71\\
\footnotesize - \>\footnotesize  ALPHARV(1) \>\footnotesize $\ldots$(2) \>\footnotesize
$\ldots$(3) \>\footnotesize $\ldots$(4) \>\footnotesize $\ldots$(5) \>\footnotesize ALPHARV(6)
\\
\end{tabbing}
\setw{ALPHARV(I) X}
\begin{tabbing}
ALPHARV(I) X\= set tab \kill
ALPHARV(I) \> \parbox[t]{\width}{The density exponent ALPHAR (see type 02 card)
at time TPRO(I) (0.5).  If included, this overwrites the value on the type 04 card.}
\end{tabbing}
\pagebreak
\subsubsection{Card 48 - Density Exponent 2}
\index{density profile}
\begin{tabbing}
XXXXXX \= XXXXXXX \= XXXXXX \= XXXXXX \= XXXXXX \= XXXXXX \=
XXXXXX \kill
11 \> 21 \> 31 \> 41 \> 51 \> 61 \> 71\\
\footnotesize - \>\footnotesize  BETARV(1) \>\footnotesize $\ldots$(2) \>\footnotesize $\ldots$(3)
\>\footnotesize $\ldots$(4) \>\footnotesize $\ldots$(5) \>\footnotesize BETARV(6) \\
\end{tabbing}
\setw{BETARV(I) X}
\begin{tabbing}
BETARV(I) X\= set tab \kill
BETARV(I) \> \parbox[t]{\width}{The density exponent BETAR (see type 02 card)
 at time TPRO(I) (2.0).  If included, this overwrites the value on the type 04 card.}
\end{tabbing}

\newpage \subsubsection{Card 49 - Multipolar Moments}
\index{multipolar moments}
\begin{tabbing}
XXXXXX \= XXXXXX \= XXXXXXX \= XXXXXXXX \= XXXXXX \= XXXXXX
\=XXXXXX \kill
11 \> 21 \> 31 \> 41 \> 51 \> 61 \> 71\\
\footnotesize N \>\footnotesize MULTN(N) \>\footnotesize ROMULT(N) \>\footnotesize
IGROUPM(N) \> \footnotesize ATURNSM(N)\\
\end{tabbing}
\setw{ATURNSM(N) = 0.0X}
\begin{tabbing}
ATURNSM(N)\= = 0.0X\= set tab \kill
N \> \> \parbox[t]{\width}{Multipole coil number (this must be a unique identifying number
between 1 and PNCOIL}\\
MULTN(N) \> \>Multipole field type :\\
\> = 0.0 \> Even nullapole\\
\> = 1.0 \> Odd nullapole\\
\> = 2.0 \> Even dipole\\
\> = 3.0 \> Odd dipole\\
\> = 4.0 \> Even quadrupole\\
\> = 5.0 \> Odd quadrupole\\
\> = 6.0 \> Even hexapole\\
\> = 7.0 \> Odd hexapole\\
\> = 8.0 \> Even Octapole\\
\> = 9.0 \> Odd Octapole\\
\> = 10.0 \> Even decapole\\
ROMULT(N) \> \> \parbox[t]{\width}{Major radius about which multipole fields are
expanded}\\
IGROUPM(N) \> \> \parbox[t]{\width}{Group number of multipole coil N. Refers to type 15
card with the same number.}\\
ATURNSM(N) \> \> \parbox[t]{\width}{Number of turns for multipole coil N. This is a positive
or negative number, not necessarily an integer.  The preprogrammed current for multipole coil
N will be the product of ATURNSC(N) and the current in IGROUPC(N) 
as specified by the appropriate type 15 card.}
\end{tabbing}

\newpage \subsubsection{Card 50 - Neutral Beam Fraction}
\index{neutral beam!orientation fraction}
\begin{tabbing}
XXXXXX \= XXXXXXXX \= XXXXXX \= XXXXXX \= XXXXXX \= XXXXXX \=
XXXXXX \kill
11 \> 21 \> 31 \> 41 \> 51 \> 61 \> 71\\
\footnotesize - \>\footnotesize  FRACPAR(1) \>\footnotesize $\ldots$(2) \>\footnotesize
$\ldots$(3) \>\footnotesize $\ldots$(4) \>\footnotesize $\ldots$(5) \>\footnotesize FRACPAR(6)
\\
\end{tabbing}
\setw{FRACPAR(I) X}
\begin{tabbing}
FRACPAR(I) X\= set tab \kill
FRACPAR(I) \> \parbox[t]{\width}{The fraction of neutral beams oriented tangentially at time
TPRO(I).  If included, this overwrites the value on the type 25 card.}
\end{tabbing}

\newpage \subsubsection{Cards 51-54 Input Power Profile (LH)}
\index{lower hybrid waves!input power}
\begin{tabbing}
XXXXXX \= XXXXXX \= XXXXXX \= XXXXXX \= XXXXXX \= XXXXXX \=XXXXXX
\kill
11 \> 21 \> 31 \> 41 \> 51 \> 61 \> 71\\
- \> \footnotesize A(1) \>\footnotesize A(2) \>\footnotesize A(3) \>\footnotesize A(4)
\>\footnotesize A(5) \>\footnotesize A(6)\\
- \> \footnotesize D(1) \>\footnotesize D(2) \>\footnotesize D(3) \>\footnotesize D(4)
\>\footnotesize D(5) \>\footnotesize D(6)\\
- \> \footnotesize A1(1) \>\footnotesize A1(2) \>\footnotesize A1(3) \>\footnotesize A1(4)
\>\footnotesize A1(5) \>\footnotesize A1(6)\\
- \>\footnotesize A2(1) \>\footnotesize A2(2) \>\footnotesize A2(3) \>\footnotesize A2(4)
\>\footnotesize A2(5) \>\footnotesize A2(6)\\
\end{tabbing}
The above cards specify the input power profile for lower hybrid waves at time TPRO according
to
\begin{equation}
S_{LH}(\hat{\Psi}) = \frac{d^2 \hat{\Psi}^{a_1}(1-\hat{\Psi})^{a_2}}{(\hat{\Psi}-a)^2+d^2} ,
\end{equation}
where $\hat{\Psi}=(\Psi-\Psi_{min})/(\Psi_{lim}-\Psi_{min})$.  Normalization is such that the total power
in MW is given on the type 46 card.

\newpage \subsubsection{Cards 55-58 Current Profile (LH)}
\index{lower hybrid waves!current profile}
\begin{tabbing}
XXXXXX \= XXXXXX \= XXXXXX \= XXXXXX \= XXXXXX \= XXXXXX \=XXXXXX
\kill
11 \> 21 \> 31 \> 41 \> 51 \> 61 \> 71\\
- \> \footnotesize AC(1) \>\footnotesize AC(2) \>\footnotesize AC(3) \>\footnotesize AC(4)
\>\footnotesize AC(5) \>\footnotesize AC(6)\\
- \> \footnotesize DC(1) \>\footnotesize DC(2) \>\footnotesize DC(3) \>\footnotesize DC(4)
\>\footnotesize DC(5) \>\footnotesize DC(6)\\
- \> \footnotesize AC1(1) \>\footnotesize AC1(2) \>\footnotesize AC1(3) \>\footnotesize AC1(4)
\>\footnotesize AC1(5) \>\footnotesize AC1(6)\\
- \> \footnotesize AC2(1) \>\footnotesize AC2(2) \>\footnotesize AC2(3) \>\footnotesize AC2(4)
\>\footnotesize AC2(5) \>\footnotesize AC2(6)\\
\end{tabbing}
The above cards specify the current profile for lower hybrid waves evolving independently in
time from the power according to
\begin{equation}
J_{LH}(\hat{\Psi}) = \frac{d^2_c \hat{\Psi}^{a_{c1}}(1-\hat{\Psi})^{a_{c2}}}{(\hat{\Psi}-a_c)^2+d_c^2} .
\end{equation}
Linear interpolation is used between different time values. Normalization is such that the total current is given by the Fisch formula.
This resulting total current can be adjusted with ACOEF(106).
\newpage \subsubsection{Card 59 - ICRH Power}
\index{ICRH}
\begin{tabbing}
XXXXXX \= XXXXXXXX \= XXXXXX \= XXXXXX \= XXXXXX \= XXXXXX \=
XXXXXX \kill
11 \> 21 \> 31 \> 41 \> 51 \> 61 \> 71\\
\footnotesize - \>\footnotesize  PICRH(1) \>\footnotesize $\ldots$(2) \>\footnotesize
$\ldots$(3) \>\footnotesize $\ldots$(4) \>\footnotesize $\ldots$(5) \>\footnotesize PICRH(6)
\\
\end{tabbing}
\setw{PICRH(I) X}
\begin{tabbing}
PICRH(I) X\= set tab \kill
PICRH(I) \> \parbox[t]{\width}{The amplitude of the ICRH source(MW) at time TPRO(I).  The
deposition profile for electron heating is given on the TYPE 65-68 cards.}
\end{tabbing}
\newpage \subsubsection{Cards 60 and 61 - Halo Parameters}
\index{feedback!systems}
\begin{tabbing}
XXXXXX \= XXXXXX \= XXXXXX \= XXXXXX \= XXXXXX \= XXXXXX \=
XXXXXX \kill
11 \> 21 \> 31 \> 41 \> 51 \> 61 \> 71\\
\footnotesize  - \>\footnotesize TH(1)  \>\footnotesize TH(2) \>\footnotesize
TH(3) \>\footnotesize TH(4) \>\footnotesize TH(5) \>\footnotesize TH(6)\\
\footnotesize  - \>\footnotesize AH(1)  \>\footnotesize AH(2) \>\footnotesize
AH(3) \>\footnotesize AH(4) \>\footnotesize AH(5) \>\footnotesize AH(6)
\end{tabbing}
\setw{TH(I) X}
\begin{tabbing}
TH(I) X\=  set tabs \kill
TH(I)  \=  \parbox[t]{\width}{The halo region temperature (in eV) corresponding to time
TPRO(I) }\\
AH(I)  \=  \parbox[t]{\width}{The halo region width (described as a
fraction of the poloidal flux between the plasma edge and axis,
($\delta \psi_{halo}=AH(I) (\psi_{edge}-\psi_{axis})$) corresponding to time
TPRO(I) }
\end{tabbing}
{\bf Note 1} :  Tedge must be set to zero, ACOEF(880) \\
{\bf Note 2} : these values override those specified in
ACOEF(97) and ACOEF(98)
{\bf Note 3} : \parbox[t]{\width}{Either the type 97 or the AH parameter must be included
for the temperature boundary condition to be applied.}

\newpage \subsubsection{Cards 62 and 63 - Control Points}
\index{feedback!systems}
\begin{tabbing}
XXXXXX \= XXXXXX \= XXXXXX \= XXXXXX \= XXXXXX \= XXXXXX \=
XXXXXX \kill
11 \> 21 \> 31 \> 41 \> 51 \> 61 \> 71\\
\footnotesize  - \>\footnotesize XCON0(1)  \>\footnotesize XCON0(2) \>\footnotesize
XCON0(3) \>\footnotesize XCON0(4) \>\footnotesize XCON0(5) \>\footnotesize XCON0(6)\\
\footnotesize  - \>\footnotesize ZCON0(1)  \>\footnotesize ZCON0(2) \>\footnotesize
ZCON0(3) \>\footnotesize ZCON0(4) \>\footnotesize ZCON0(5) \>\footnotesize ZCON0(6)
\end{tabbing}
\setw{XCON0(I) X}
\begin{tabbing}
XCON0(I) X\=  set tabs \kill
XCON0(I)  \=  \parbox[t]{\width}{The $x$-shape point corresponding to time
TPRO(I) for use in equilibrium iteration when acoef(901)$>$0
and shape control (IPEXT=25 on type 20 card)}\\
ZCON0(I)  \=  \parbox[t]{\width}{The $z$-shape point corresponding to time
TPRO(I) for use in equilibrium iteration when acoef(901)$>$0
and shape control (IPEXT=25 on type 20 card)}
\end{tabbing}
\newpage \subsubsection{Card 64 - ICRH Fast Wave Current}
\index{ICRH}
\begin{tabbing}
XXXXXX \= XXXXXXXX \= XXXXXX \= XXXXXX \= XXXXXX \= XXXXXX \=
XXXXXX \kill
11 \> 21 \> 31 \> 41 \> 51 \> 61 \> 71\\
\footnotesize - \>\footnotesize  FWCD(1) \>\footnotesize $\ldots$(2) \>\footnotesize
$\ldots$(3) \>\footnotesize $\ldots$(4) \>\footnotesize $\ldots$(5) \>\footnotesize FWCD(6)
\\
\end{tabbing}
\setw{FWCD(I) X}
\begin{tabbing}
FWCD(I) X\= set tab \kill
FWCD(I) \> \parbox[t]{\width}{The total toroidal current(MA) driven by fast wave current
drive at time TPRO(I).  The deposition profile is given on the TYPE 69-72 cards.}
\end{tabbing}
\newpage \subsubsection{Cards 65-68 Input Power Profile(ICRH)}
\index{ICRH heating}
\begin{tabbing}
XXXXXX \= XXXXXX \= XXXXXX \= XXXXXX \= XXXXXX \= XXXXXX \=XXXXXX
\kill
11 \> 21 \> 31 \> 41 \> 51 \> 61 \> 71\\
- \> \footnotesize A(1) \>\footnotesize A(2) \>\footnotesize A(3) \>\footnotesize A(4)
\>\footnotesize A(5) \>\footnotesize A(6)\\
- \> \footnotesize D(1) \>\footnotesize D(2) \>\footnotesize D(3) \>\footnotesize D(4)
\>\footnotesize D(5) \>\footnotesize D(6)\\
- \> \footnotesize A1(1) \>\footnotesize A1(2) \>\footnotesize A1(3) \>\footnotesize A1(4)
\>\footnotesize A1(5) \>\footnotesize A1(6)\\
- \> \footnotesize A2(1) \>\footnotesize A2(2) \>\footnotesize A2(3) \>\footnotesize A2(4)
\>\footnotesize A2(5) \>\footnotesize A2(6)\\
\end{tabbing}
The above cards specify the input power profile for electrons for ICRH heating.
The power density from fast wave at time t is given by:
\begin{equation}
S_{ICRH}(\hat{\Psi},t) ={\alpha}_N(t) \frac{d^2 \hat{\Psi}^{a_{1}}(1-\hat{\Psi})^{a_{2}}}{(\hat{\Psi}-a)^2+d^2} .
\end{equation}
where $\hat{\Psi}=( \Psi - {\Psi}_{min}) / ({\Psi}_{lim}-{\Psi}_{min})$.  The normalization parameter
${\alpha}_N(t)$ is chosen such that the total power from fast waves in MW is given on the type 59 card.
\newpage \subsubsection{Cards 69-72 Current Profile (FW)}
\index{ICRH fast waves!current profile}
\begin{tabbing}
XXXXXX \= XXXXXX \= XXXXXX \= XXXXXX \= XXXXXX \= XXXXXX \=XXXXXX
\kill
11 \> 21 \> 31 \> 41 \> 51 \> 61 \> 71\\
- \> \footnotesize A(1) \>\footnotesize A(2) \>\footnotesize A(3) \>\footnotesize A(4)
\>\footnotesize A(5) \>\footnotesize A(6)\\
- \> \footnotesize D(1) \>\footnotesize D(2) \>\footnotesize D(3) \>\footnotesize D(4)
\>\footnotesize D(5) \>\footnotesize D(6)\\
- \> \footnotesize A1(1) \>\footnotesize A1(2) \>\footnotesize A1(3) \>\footnotesize A1(4)
\>\footnotesize A1(5) \>\footnotesize A1(6)\\
- \> \footnotesize A2(1) \>\footnotesize A2(2) \>\footnotesize A2(3) \>\footnotesize A2(4)
\>\footnotesize A2(5) \>\footnotesize A2(6)\\
\end{tabbing}
The above cards specify the input current density profile for fast wave current drive.
The current density from fast wave at time t is given by:
\begin{equation}
\vec{J_{FW}}= {\alpha}_{N}(t)f(\hat{\Psi},t)\vec{B}
\end{equation}
where
\begin{equation}
f(\hat{\Psi},t) = \frac{d^2 \hat{\Psi}^{a_{1}}(1-\hat{\Psi})^{a_{2}}}{(\hat{\Psi}-a)^2+d^2} .
\end{equation}
where $\hat{\Psi}=( \Psi - {\Psi}_{min}) / ({\Psi}_{lim}-{\Psi}_{min})$.  The normalization parameter
${\alpha}_N(t)$ is chosen such that the total current driven by fast waves in MA is given on the type 64 card.
\newpage \subsubsection{Card 73 - He Confinement Time}
\index{He confinement time}
\begin{tabbing}
XXXXXX \= XXXXXXXX \= XXXXXX \= XXXXXX \= XXXXXX \= XXXXXX \=
XXXXXX \kill
11 \> 21 \> 31 \> 41 \> 51 \> 61 \> 71\\
\footnotesize - \>\footnotesize  HEACT(1) \>\footnotesize $\ldots$(2) \>\footnotesize
$\ldots$(3) \>\footnotesize $\ldots$(4) \>\footnotesize $\ldots$(5) \>\footnotesize HEACT(6)
\\
\end{tabbing}
\setw{HEACT(I) X}
\begin{tabbing}
HEACT(I) X\= set tab \kill
HEACT(I) \> \parbox[t]{\width}{The He confinement time in seconds
at time TPRO(I).  This overrides the value given in ACOEF(93).}
\end{tabbing}
\newpage \subsubsection{Card 74 - UFILE Output}
\index{UFILE}
\begin{tabbing}
XXXXXX \= XXXXXXXX \= XXXXXX \= XXXXXX \= XXXXXX \= XXXXXX \=
XXXXXX \kill
11 \> 21 \> 31 \> 41 \> 51 \> 61 \> 71\\
\footnotesize - \>\footnotesize  TUFILE(1) \>\footnotesize $\ldots$(2) \>\footnotesize
$\ldots$(3) \>\footnotesize $\ldots$(4) \>\footnotesize $\ldots$(5) \>\footnotesize TUFILE(6)
\\
\end{tabbing}
\setw{TUFILE(I) X}
\begin{tabbing}
TUFILE(I) X\= set tab \kill
TUFILE(I) \> \parbox[t]{\width}{Specified times at which UFILE output
is to be written into file OGRAPH for ACOEF(3001)=1.}
\end{tabbing}
\newpage \subsubsection{Card 75 - SAWTOOTH times}
\index{UFILE}
\begin{tabbing}
XXXXXX \= XXXXXXXX \= XXXXXX \= XXXXXX \= XXXXXX \= XXXXXX \=
XXXXXX \kill
11 \> 21 \> 31 \> 41 \> 51 \> 61 \> 71\\
\footnotesize - \>\footnotesize  TUFILE(1) \>\footnotesize $\ldots$(2) \>\footnotesize
$\ldots$(3) \>\footnotesize $\ldots$(4) \>\footnotesize $\ldots$(5) \>\footnotesize TUFILE(6)
\\
\end{tabbing}
\setw{SAWTIME(I) X}
\begin{tabbing}
SAWTIME(I) X\= set tab \kill
SAWTIME(I) \> \parbox[t]{\width}{Times (in seconds) for which sawtooth will
occur for ISAW=2 on type 12}
\end{tabbing}
\pagebreak
\subsubsection{Card 76 - Anomalous Ion Transport for itrmod=13,14}
\index{transport!anomalous}
\begin{tabbing}
XXXXX \= XXXXXXX \= XXXXXXX \= XXXXXXX \= XXXXXXX \= XXXXXXX \=
XXXXXX \kill
11 \> 21 \> 31 \> 41 \> 51 \> 61 \> 71\\
\footnotesize  - \>\footnotesize FBCHIIA(1)  \>\footnotesize FBCHIIA(2) \>\footnotesize
FBCHIIA(3) \>\footnotesize FBCHIIA(4) \>\footnotesize FBCHIIA(5) \>\footnotesize FBCHIIA(6)
\end{tabbing}
\setw{FBCHIIA(I) X}
\begin{tabbing}
FBCHIIA(I) X\= \parbox[t]{\width}{This is junk to set tabs} \kill
FBCHIIA(I) \>\parbox[t]{\width}{Factor by which ion thermal conductivity (for itrmod=13,14) is enhanced at time TPRO(I)} \\
\end{tabbing}

\newpage \subsubsection{Cards 77-89 new time-dependent variables (updated Dec-20-2010)}
\index{new time-dependent variables}
\begin{tabbing}
XXXXXX \= XXXXXX \= XXXXXX \= XXXXXX \= XXXXXX \= XXXXXX \=XXXXXX
\kill
11 \> 21 \> 31 \> 41 \> 51 \> 61 \> 71\\
- \> \footnotesize qadd(1) \>\footnotesize qadd(2) \>\footnotesize qadd(3) \>\footnotesize qadd(4)
\>\footnotesize qadd(5) \>\footnotesize qadd(6)\\
- \> \footnotesize fhmodei(1) \>\footnotesize fhmodei(2) \>\footnotesize fhmodei(3) \>\footnotesize fhmodei(4)
\>\footnotesize fhmodei(5) \>\footnotesize fhmodei(6)\\
- \> \footnotesize pwidthc(1) \>\footnotesize pwidthc(2) \>\footnotesize pwidthc(3) \>\footnotesize pwidthc(4)
\>\footnotesize pwidthc(5) \>\footnotesize pwidthc(6)\\
- \> \footnotesize chiped(1) \>\footnotesize chiped(2) \>\footnotesize chiped(3) \>\footnotesize chiped(4)
\>\footnotesize chiped(5) \>\footnotesize chiped(6)\\
- \> \footnotesize tped(1) \>\footnotesize tped(2) \>\footnotesize tped(3) \>\footnotesize tped(4)
\>\footnotesize tped(5) \>\footnotesize tped(6)\\
imptype \> \footnotesize frac(1) \>\footnotesize frac(2) \>\footnotesize frac(3) \>\footnotesize frac(4)
\>\footnotesize frac(5) \>\footnotesize frac(6)\\
- \> \footnotesize nflag(1) \>\footnotesize nflag(2) \>\footnotesize nflag(3) \>\footnotesize nflag(4)
\>\footnotesize nflag(5) \>\footnotesize nflag(6)\\
- \> \footnotesize expn1(1) \>\footnotesize expn1(2) \>\footnotesize expn1(3) \>\footnotesize expn1(4)
\>\footnotesize expn1(5) \>\footnotesize expn1(6)\\
- \> \footnotesize expn2(1) \>\footnotesize expn2(2) \>\footnotesize expn2(3) \>\footnotesize expn2(4)
\>\footnotesize expn2(5) \>\footnotesize expn2(6)\\
- \> \footnotesize firitb(1) \>\footnotesize firitb(2) \>\footnotesize firitb(3) \>\footnotesize firitb(4)
\>\footnotesize firitb(5) \>\footnotesize firitb(6)\\
- \> \footnotesize secitb(1) \>\footnotesize secitb(2) \>\footnotesize secitb(3) \>\footnotesize secitb(4)
\>\footnotesize secitb(5) \>\footnotesize secitb(6)\\
- \> \footnotesize fracn0(1) \>\footnotesize fracn0(2) \>\footnotesize fracn0(3) \>\footnotesize fracn0(4)
\>\footnotesize fracn0(5) \>\footnotesize fracn0(6)\\
- \> \footnotesize newden(1) \>\footnotesize newden(2) \>\footnotesize newden(3) \>\footnotesize newden(4)
\>\footnotesize newden(5) \>\footnotesize newden(6)\\
\end{tabbing} 
\begin{tabbing}
77 X\= set tab \kill
77 \> ------   qadd(1)    qadd(2)  ....  qadd(6)  (default acoef(123)) \\
78 X\= set tab \kill
78 \> ------   fhmodei(1) fhmodei(2) ..  fhmodei(6) (default acoef(3003)) \\
79 X\= set tab \kill
79 \> ------   pwidthc(1) pwidthc(2) ..  pwidthc(6) (default acoef(3011)) \\
80 X\= set tab \kill
80 \> ------   chiped(1)  chiped(2)  ..  chiped(6)  (default acoef(3006)) \\
81 X\= set tab \kill
81 \> ------   tped(1)    tped(2)    ..  tped(6)    (default acoef(3102)) \\
82 X\= set tab \kill
82 \> imptype   frac(1)    frac(2)  ....  frac(6) (default acoef(853+imptype) \\
83 X\= set tab \kill
83 \> ------   nflag(1)    nflag(2)  ....  nflag(6)  (default acoef(3012)) \\
84 X\= set tab \kill
84 \> ------   expn1(1)    expn1(2)  ....  expn1(6)  (default acoef(3013)) \\
85 X\= set tab \kill
85 \> ------   expn2(1)    expn2(2)  ....  expn2(6)  (default acoef(3014)) \\
86 X\= set tab \kill
86 \> ------   firitb(1)    firitb(2)  ....  firitb(6)  (default acoef(3004)) \\
87 X\= set tab \kill
87 \> ------   secitb(1)    secitb(2)  ....  secitb(6)  (default acoef(3005)) \\
88 X\= set tab \kill
88 \> ------   fracn0(1)    fracn0(2)  ....  fracn0(6)  (default acoef(881)) \\
89 X\= set tab \kill
89 \> ------   newden(1)    newden(2)  ....  newden(6)  (default acoef(889)) \\
\end{tabbing}


%cj jan 28, 2011
\newpage \subsubsection{Card 90 - ECRH Power (MW)}
\index{PECRH}
\begin{tabbing}
XXXXXX \= XXXXXXXX \= XXXXXX \= XXXXXX \= XXXXXX \= XXXXXX \=
XXXXXX \kill
11 \> 21 \> 31 \> 41 \> 51 \> 61 \> 71\\
\footnotesize - \>\footnotesize  PECRH(1) \>\footnotesize $\ldots$(2) \>\footnotesize
$\ldots$(3) \>\footnotesize $\ldots$(4) \>\footnotesize $\ldots$(5) \>\footnotesize PECRH(6)
\\
\end{tabbing}
\setw{FWCD(I) X}
\begin{tabbing}
PECRH(I) X\= set tab \kill
FWCD(I) \> \parbox[t]{\width}{
ECRH Power (MW)
}
\end{tabbing}

\subsubsection{Card 91 - ECCD Toroidal Current (MA)}
\index{ECCD}
\begin{tabbing}
XXXXXX \= XXXXXXXX \= XXXXXX \= XXXXXX \= XXXXXX \= XXXXXX \=
XXXXXX \kill
11 \> 21 \> 31 \> 41 \> 51 \> 61 \> 71\\
\footnotesize - \>\footnotesize  ECCD(1) \>\footnotesize $\ldots$(2) \>\footnotesize
$\ldots$(3) \>\footnotesize $\ldots$(4) \>\footnotesize $\ldots$(5) \>\footnotesize ECCD(6)
\\
\end{tabbing}
\setw{ECCD(I) X}
\begin{tabbing}
ECCD(I) X\= set tab \kill
ECCD(I) \> \parbox[t]{\width}{
ECCD Toroidal Current (MA)
}
\end{tabbing}

\subsubsection{Card 92 -  First shape parameter “a” for ECCD heating AND CD}
\index{AECD}
\begin{tabbing}
XXXXXX \= XXXXXXXX \= XXXXXX \= XXXXXX \= XXXXXX \= XXXXXX \=
XXXXXX \kill
11 \> 21 \> 31 \> 41 \> 51 \> 61 \> 71\\
\footnotesize - \>\footnotesize  AECD(1) \>\footnotesize $\ldots$(2) \>\footnotesize
$\ldots$(3) \>\footnotesize $\ldots$(4) \>\footnotesize $\ldots$(5) \>\footnotesize AECD(6)
\\
\end{tabbing}
\setw{AECD(I) X}
\begin{tabbing}
AECD(I) X\= set tab \kill
AECD(I) \> \parbox[t]{\width}{
First shape parameter “a” for ECCD heating AND CD
}
\end{tabbing}

\subsubsection{Card 93 -  Second shape parameter “d” for ECCD heating AND CD}
\index{DECD}
\begin{tabbing}
XXXXXX \= XXXXXXXX \= XXXXXX \= XXXXXX \= XXXXXX \= XXXXXX \=
XXXXXX \kill
11 \> 21 \> 31 \> 41 \> 51 \> 61 \> 71\\
\footnotesize - \>\footnotesize  DECD(1) \>\footnotesize $\ldots$(2) \>\footnotesize
$\ldots$(3) \>\footnotesize $\ldots$(4) \>\footnotesize $\ldots$(5) \>\footnotesize DECD(6)
\\
\end{tabbing}
\setw{DECD(I) X}
\begin{tabbing}
DECD(I) X\= set tab \kill
DECD(I) \> \parbox[t]{\width}{
Second shape parameter “d” for ECCD heating AND CD
}
\end{tabbing}

\subsubsection{Card 94 -  Third shape parameter “a1” for ECCD heating AND CD}
\index{A1ECD}
\begin{tabbing}
XXXXXX \= XXXXXXXX \= XXXXXX \= XXXXXX \= XXXXXX \= XXXXXX \=
XXXXXX \kill
11 \> 21 \> 31 \> 41 \> 51 \> 61 \> 71\\
\footnotesize - \>\footnotesize  A1ECD(1) \>\footnotesize $\ldots$(2) \>\footnotesize
$\ldots$(3) \>\footnotesize $\ldots$(4) \>\footnotesize $\ldots$(5) \>\footnotesize A1ECD(6)
\\
\end{tabbing}
\setw{A1ECD(I) X}
\begin{tabbing}
A1ECD(I) X\= set tab \kill
A1ECD(I) \> \parbox[t]{\width}{
Third shape parameter “a1” for ECCD heating AND CD
}
\end{tabbing}

\subsubsection{Card 95 -  Fourth shape parameter “a2” for ECCD heating AND CD}
\index{A2ECD}
\begin{tabbing}
XXXXXX \= XXXXXXXX \= XXXXXX \= XXXXXX \= XXXXXX \= XXXXXX \=
XXXXXX \kill
11 \> 21 \> 31 \> 41 \> 51 \> 61 \> 71\\
\footnotesize - \>\footnotesize  A2ECD(1) \>\footnotesize $\ldots$(2) \>\footnotesize
$\ldots$(3) \>\footnotesize $\ldots$(4) \>\footnotesize $\ldots$(5) \>\footnotesize A2ECD(6)
\\
\end{tabbing}
\setw{A2ECD(I) X}
\begin{tabbing}
A2ECD(I) X\= set tab \kill
A2ECD(I) \> \parbox[t]{\width}{
Third shape parameter “a2” for ECCD heating AND CD
} 
\end{tabbing}

\end{document}
